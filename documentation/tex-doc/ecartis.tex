\documentclass{book}
\usepackage{ecartis}
\usepackage{hthtml}
\usepackage{verbatim}
\usepackage{longtable}

\begin{document}

\frontmatter

\title{Ecartis \\ Modular Mailing List Manager \\ http://www.ecartis.net/}
\author{Copyright \copyright\ 1998--2002 Rachel Blackman, JT Traub and contributors.}
\maketitle

\chapter{Version History}
\label{version}

\begin{description}
	\item[2002-04-29] Rewrite from Listar docs to Ecartis docs; added
                      paragraph to Introduction about name change from Listar
                      to Ecartis
	\item[2016-07-04] Changed Domain name. ecartis.org -\textgreater  ecartis.net
	\item[2016-07-04] Added appendix A - Config Variables
\end{description}

\chapter{Notes About This Document}

The Ecartis manual is most definitely a work in progress.  As is common with
many software projects, development of the software has far exceeded
development of the documentation to explain it; this is a shortcoming we are
attempting to address.  

Discussion of this documentation should be directed to the mailing list 
\htmailto{ecartis-doc@ecartis.net}.  Subscription information for the list is
available at \hturl{http://www.ecartis.net}.  Anyone who has submissions they
would like added to the documentation, or has suggestions for rewording,
changes, etc. to the existing documentation should direct their comments to
this list.

For purposes of portability, this documentation is currently maintained in
\LaTeX, a very simple, yet powerful text markup language that provides us the
ability to easily generate versions of this documentation formatted as
PostScript, PDF, HTML, or raw text.  Those wishing to make direct changes to
the documentation source should first familiarize themselves with \LaTeXe,
and should keep the ideal of portability in mind when making decisions
regarding format.



\tableofcontents



\mainmatter


\chapter{Introduction}
\label{int}
\section{What is a Mailing List Manager?}
\label{int:whatis}

A mailing list manager is a piece of computer software which accepts a piece
of e-mail from a single source and distributes it to a number of recipients.
Possible uses for such a piece of software include a monthly newsletter for
customers of a business, a way to distribute information to users of a
particular software package --- for example, notification of a security fix
--- or a way for people who share a common interest to communicate with each
other.
   
How people choose to implement mailing lists can vary widely.  The simplest
method involves setting up a single address on an e-mail domain you control,
and forwarding it to a number of people.  This has a number of disadvantages,
however; users have no way to add themselves to the list, or remove themselves
from it, there are no ways to restrict who can send mail to a distribution
list, and other such headaches.
   
Most people, therefore, choose to use a piece of software to manage such lists
for them; hence the term mailing list manager, or MLM for short.  Many people
also refer to such packages as `list-servers,' as they are server software for
managing lists.  There are an ever-increasing number of MLMs available out
there, but almost all share certain common traits; the ability for users to
subscribe or unsubscribe themselves from a distribution list, the ability for
an administrator to manually remove a user, the ability to restrict posting to
a small number of individuals, and so on.  In addition, some MLMs support many
additional advanced features, such as the ability to filter out unsolicited
commercial e-mail (UCE, also known as `spam').
   
\section{A Brief History of MLMs}
\label{int:history}

Back in the mid-1980's, the system of interconnected computers we know as the
Internet was not yet around.  While in the United States, there was some
interconnection between colleges and the government's ARPAnet, the only way
any other machines --- such as those at different universities ---
communicated was over a system called BITNET.  BITNET machines
let messages for each other pile up, and then would call each other over the
phone lines and send the messages.

BITNET had a central control post, a Network Information Center (or NIC)
called `BITNIC.'  BITNIC kept a number of distribution lists for BITNET users.
However, the BITNIC's lists were set up in the primitive way mentioned in
Section~\ref{int:whatis}; a single address with no way for users to add
themselves or remove themselves.  If you wished to be on a mailing list, you
had to contact the BITNIC staff and have them add you by hand.

Unfortunately, as BITNET grew larger, managing the lists by hand was no longer
feasible.  Additionally, since all mail for BITNIC was affected by the traffic
of the lists --- which by now were quite large --- even private BITNET e-mail
was affected.  It may be hard for users of today's Internet to imagine, but
try to picture things becoming so slow that when someone sent you an e-mail,
it took over a week to arrive in your mailbox.  Clearly, something needed to
be done.
   
Since the source of the problem was the traffic on BITNIC's mailing lists, a
computer science student named Eric Thomas decided to write a piece of
software to replace the manually-managed mailing lists.  It also used a number
of alternate paths to send e-mail to the list recipients, to keep it from
clogging BITNET's mail pathways, but what was more important to the future of
MLMs was the fact that this software allowed users to add and remove
themselves from the lists, instead of relying on a BITNIC system administrator
to do it for them.  When it went online in July 1986, a piece of software was
managing a mailing list for the first time --- Eric Thomas had created the first
MLM.  Soon thereafter, others created similar packages for other systems, such
as the LISTSERV imitation for UNIX, Listproc.
   
As time went on, Eric Thomas' LISTSERV developed into a commercial product
beyond BITNET and was ported to other systems, and is still widely used on the
Internet today.  However, as the days of BITNET faded into the past and the
Internet became a reality, students who wished to run their own lists and did
not have access to the funds necessary to purchase a license for LISTSERV
began to look at developing their own MLMs to meet their particular needs.
   
Perhaps one of the most popular is Majordomo, which has been
worked on by a variety of people over the years.  Majordomo is written in the
Perl scripting language, which is perhaps its greatest failing as it makes
Majordomo rather inefficient.  However, Perl is very powerful for text
processing, and thus Majordomo is readily extendible by those who know Perl
and are willing to learn Majordomo's source code.
   
A good --- if somewhat biased --- summary of the history of MLMs is available
online from Lyris Technologies (who are themselves the authors of a commercial
MLM called Lyris, which is targeted specifically at business users) at
\hturl{http://www.lyristechnologies.com/historyls.html}
   
\section{History of Ecartis}
\label{int:history:Ecartis}

Ecartis was born as a simple little project called `uList' (microList) in
October 1997.  The original author, Rachel Blackman, had been using the
Majordomo mailing list package but wished for one that was more efficient and
did not require any special system permissions to run.  Additionally, Rachel
wanted a server that would allow the individual subscribers to change their
subscription options --- one of the more desirable features of LISTSERV.
   
The original design for uList was simple enough; users needed to be able to
set a few simple options on themselves as well as perform all the standard
operations that most MLMs provide.  However, Rachel got tired of having to
constantly rewrite the core processing code as new functionality was added,
and changed uList over to a modular system, where almost all of the system was
added to a changeable table of information.  Suddenly, the system could be
easily changed; a new command could be added with only a few lines of code, or
a new step in processing a message to be sent.  The small uList program
suddenly had more potential.  And to go with the new design, the software got
a new name: Listar.
   
Listar developed into a more stable piece of software, and a mailing list was
set up using it called listar-dev, for those who were interested in the
ongoing development of the project.  On January 12, 1998, Joseph (JT) Traub
began to work on the project as well.  JT's first contribution to the project
was the development of a `dynamic module' system.  Since Listar was already
based entirely around a dynamic model, this allowed new Listar plugin modules
(or LPMs) to be installed and immediately provide new commands, subscription
flags, or functionality.
   
The first public release of Listar came in February of 1998, and it was used
only by a few curious parties.  However, many provided good feedback on
features and functionality they wished to see in such a project, and Listar
grew rapidly into a more mature program.
   
Then, tragedy struck.  In October of 1998, the machine that Listar's main
development resources were housed on was cracked into by a malicious
individual, and the mailing lists were destroyed.  The Listar source code
remained safe in backup copies, but the lists themselves were no longer
available.  The recovery from this event took a while, and development on
Listar was slow again afterwards at first, until users discovered the site was
back and resubscribed to the lists.

Once past the recovery, however, 1999 proved to be a year full of rapid
development for Listar.  It became even further fleshed out, and began to be
used by some large organizations, such as the Internet Software Consortium.
The developers eagerly accepted suggestions and created new LPMs to add custom
functionality, while folding additional functionality into the core module.

Listar encountered one other setback late in 2000 (again in October) when
Rachel received a Cease \& Desist letter from the company holding a trademark
for ListSTAR, a defunct MLM for Apple computers.  Despite the obvious
differences in the names (Listar being ``List'' in Spanish; ListSTAR being
List + Star) no agreement could be reached, and in mid-2001 a name change for
the project was announced: Ecartis.  

As of this writing, Ecartis approaches the fifth anniversary of its birth as
uList, and has a growing user base as well as a budding external developer
community.  Some of the features that appeared first in Ecartis are now being
borrowed for other MLMs.  Ecartis is an Open Source project, so users are
encouraged to join in creating code for it.
   
\chapter{Installation}
\label{install}

\section{Introduction}
\label{install:intro}

There are several ways that Ecartis can be obtained, and thus several methods
of installation.  Perhaps the most common is as a collection of source code;
which will compile on any number of UNIX-type platforms, as well as Microsoft
Windows 95/98 or NT.
   
The package can also be obtained as an RPM (Redhat Package Manager)
installation, either from the Ecartis FTP site, or from RedHat's PowerTools
collection or the RedHat Contrib|Net.  It can also be obtained as a .deb file
for Debian GNU/Linux from the Ecartis FTP site or a local Debian mirror.  Or
lastly, for those more inclined towards living on the edge of development, the
source repository for Ecartis is available for read via CVS.
   
This document makes the assumption that if you choose to download either the
.rpm or .deb versions of Ecartis to install, you already know how to use that
package manager to install it, and will thus only cover compiling from source
tarball, and compiling under Windows.

\section{Source Tarball Installation}
\label{install:tarball}

The Ecartis source tarball will unpack into a directory called ecartis-version,
where version is the version of the tarball you are unpacking.  Fairly
straightforward.  Feel free to explore the directory tree before moving along
to the installation instructions for Unix (Section~\ref{install:unix}) or
Windows (Section~\ref{install:windows}).

\section{CVS Installation}
\label{install:CVS}

CVS stands for `Concurrent Versions System,' and is a method of allowing
multiple programmers to work on source code simultaneously, from a single
repository of source code. The Ecartis installation is designed to be able to
work from a CVS installation, to make it easy to keep an installation
up-to-date.  To access the Ecartis CVS tree, you need a CVS client capable of
CVS-pserver access.
   
Find the location that you wish to have your `ecartis' tree, and enter the
following command:

\begin{quote}
\footnotesize
\begin{verbatim}
cvs -d :pserver:guest@cvs.ecartis.net:/usr/cvsroot login
\end{verbatim}
\end{quote}

When it prompts you for a password, enter guest.  Then type:
   
\begin{quote}
\footnotesize
\begin{verbatim}
cvs -d :pserver:guest@cvs.ecartis.net:/usr/cvsroot checkout ecartis
\end{verbatim}
\end{quote}
   
You will see a list of filenames scroll by as CVS retrieves the current Ecartis
source tree, and then you will have a source tree as if you'd unpacked it from
the source tarball, except that it has a `CVS' subdirectory in each directory;
this allows your installation to remain synched with CVS.  Now, any time you
wish to get the latest version, go into the Ecartis directory and do:
   
\begin{quote}
\footnotesize
\begin{verbatim}
cvs update -P -d
\end{verbatim}
\end{quote}

For information on CVS and CVS servers, visit Cyclic Software, the authors of
CVS, at \hturl{http://www.cyclic.com/} --- there are graphical CVS clients
available for several operating systems as well as binaries for various
systems.
   
\section{Compiling Under UNIX}
\label{install:unix}

To compile under a UNIX-style system (such as Linux, FreeBSD, SunOS, etc.),
Ecartis needs GCC or EGCS.  While it is technically possible that Ecartis would
compile under the GCC for Windows that Cygnus provides, it is unlikely.
Ecartis has a specific Windows port; see Section~\ref{install:windows} for more
information.

The first step under UNIX is to go into the `src' directory under where you
unpacked the tarball (or did a CVS checkout to).  Copy Makefile.dist to
Makefile.  Now you'll want to pull Makefile up in your favorite text editor.
There are a variety of comments about what you'll need to tune for specific
operating systems; tweak the file as appropriate for your operating system and
then save it.  Now type `make'.
   
Ecartis uses the GNU Make program to handle the build process.  Some operating
systems, like Linux, provide this as the default `make' program, while others,
like FreeBSD, have it available as `gmake'.  If you have BSD Make, when you
run `make' in the Ecartis src directory, you will get a string of errors that
look like:

\begin{quote}
\footnotesize
\begin{verbatim}
"Makefile", line 114: Need an operator
"Makefile", line 116: Need an operator
"Makefile", line 118: Need an operator
"Makefile", line 129: Need an operator
\end{verbatim}
\end{quote}

If this is the case, you will need to run `gmake' instead of `make'.  If the
build process is going correctly, you will see output like this:
   
\begin{quote}
\footnotesize
\begin{verbatim}
gmake[1]: Entering directory `/home/loki/src/ecartis/src/modules/bounc
er'
gcc -fPIC -DDYNMOD -Wall -Werror -I../../inc -I. -DGNU_STRFTIME   -c 
bouncer.c
gcc -shared   -o bouncer.lpm bouncer.o
cp bouncer.lpm ../../build
gmake[1]: Leaving directory `/home/loki/src/ecartis/src/modules/bounce
r'
[Build: bouncer] Built module successfully.
gmake[1]: Entering directory `/home/loki/src/ecartis/src/modules/lista
rchive'
gcc -fPIC -DDYNMOD -Wall -Werror -I../../inc -I. -DGNU_STRFTIME   -c 
archive.c
gcc -shared   -o ecartischive.lpm archive.o
cp ecartischive.lpm ../../build
gmake[1]: Leaving directory `/home/loki/src/ecartis/src/modules/ecartis
chive'
[Build: ecartis] Built module successfully.
\end{verbatim}
\end{quote}

\ldots and so on.  The actual text of your output may vary slightly depending
on the system.  Assuming no build errors occur, you should end up with a final
line that looks something like this:
   
\begin{quote} 
\footnotesize
\begin{verbatim}
gcc   -o ecartis alias.o command.o user.o parse.o list.o core.o forms.
o smtp.o io.o regerror.o regsub.o regexp.o flag.o cookie.o file.o mod
ule.o fileapi.o variables.o internal.o cmdarg.o modes.o dynmod.o unmi
me.o codes.o hooks.o tolist.o mystring.o lma.o userstat.o snprintf.o 
moderate.o mysignal.o unhtml.o liscript.o submodes.o lcgi.o upgrade.o
\end{verbatim}
\end{quote}

   
If there are no linker errors, you now have all the binaries you need.  Type
`make install' (or `gmake install') and it will place all the Ecartis binaries
into the appropriate places.  If you wish to run Ecartis directly from this
location, it is now ready to go.  Otherwise, you need to copy things
elsewhere.  If you are using the dynamic module mode (Ecartis's recommended
configuration), there is a directory under where you unpacked Ecartis which is
called `modules', and contains a number of files with an .LPM extension.  And
in the directory where you unpacked Ecartis is a binary executable called
ecartis.  The ecartis binary should be placed wherever you intend to have your
Ecartis installation, along with the ecartis.cfg file.  The modules should be
placed into a modules directory, which by default is a subdirectory called
modules under wherever the ecartis binary is.  It is possible to split up
Ecartis into a number of separate directories, however, as the Debian
installation does.  At this point, you should have enough information to move
along to Section~\ref{starting}, Getting Started.
   
\section{Compiling Under Windows}
\label{install:windows}

Ecartis can be compiled under Windows, but there are a few caveats.  First, it
still functions as a mail filter, so you need a way to feed the mail to it;
many Windows mail servers do not function in this way.  Secondly, it requires
Microsoft Visual C++ 5.0 or higher to compile.

If these still meet your criteria, Ecartis is fairly easily compiled.  When you
unpack the source tree, or do a CVS checkout, you will want to go to the
top-level directory, and load the ecartis.dsw file.  Then you can simply go and
do a batch build of all the targets.  In the end, you will have a debug and a
release ecartis.exe, in \verb+src\debug\Ecartis.exe+ and
\verb+src\release\Ecartis.exe+.  You will also have \verb+src\modules\debug+
and \verb+src\modules\release+, which will contain debug and release builds of
all the LPM modules.  From here on out, Ecartis is configured like any of the
UNIX installations.
   
There is also a commercially supported mailing list package for Windows which
is based on Ecartis.  It is called SLList and is sold and supported by Seattle
Lab, at \hturl{http://www.seattlelab.com/}.

\chapter{Getting Started}
\label{starting}
   
\section{Introduction}
\label{starting:intro}

Most MLMs function as a mail filter --- in other words, a program that is
invoked when a piece of mail arrives for a specific address, and which is fed
the mail that arrived.  How individual mail servers function with this varies;
the widely-used package sendmail uses a file called aliases to determine
things like this, while other packages such as qmail handle it differently.
We'll cover a few generic pieces, and then a few specific mail servers.
   
\section{Ecartis and Mail Filters}
\label{starting:filters}

A list running on Ecartis has several addresses associated with it.  The list
address itself (for example, foolist@domain.com), the list administrators (for
example, foolist-admins@domain.com), and so on.  In addition, Ecartis has an
address for itself (ecartis@mydomain.com).  Each of these addresses is
represented by an alias.  The ecartis@mydomain.com address calls the Ecartis
program without any parameters, while the individual list ones have various
parameters.
   
Hence, when you set up the Ecartis aliases for your system, you'll probably
have a number of things looking like this:
   
\begin{quote}
\footnotesize
\begin{verbatim}
# Ecartis address
ecartis:           |/home/ecartis/ecartis
# Testlist aliases
testlist:         |"/home/ecartis/ecartis -s testlist"
testlist-request: |"/home/ecartis/ecartis -r testlist"
testlist-admins:  |"/home/ecartis/ecartis -admins testlist"
testlist-repost:  |"/home/ecartis/ecartis -a testlist"
testlist-bounce:  |"/home/ecartis/ecartis -bounce testlist"
\end{verbatim}
\end{quote}

This means that when some e-mail comes in for the `ecartis' address, the
mailserver would run the program /home/ecartis/ecartis and pass the contents of
that e-mail to the program.  The `testlist' address, for example, will also
run the program, but it will have some additional parameters.  If the aliases
look a bit confusing, don't worry; Ecartis will generate them for you when you
make a new list, but if your mailserver isn't sendmail, you may have to edit
the way the aliases look slightly.  Most mailservers use an alias format
similar --- or identical --- to the sendmail format, so for most servers you
will be able to use the aliases that Ecartis generates directly.
   
\subsection{Sendmail}
\label{starting:filters:sendmail}
   
Sendmail (\hturl{http://www.sendmail.org}) is perhaps the widest used mail
server on the Internet, and comes preinstalled on most UNIX-type systems.
Sendmail is extremely configurable --- almost too configurable, as it is
possible to get lost in the configuration options --- and is more than capable
of using Ecartis as a mail filter.
   
There are several ways to handle setting up Ecartis under Sendmail.  Perhaps
the easiest is simply to copy and paste the aliases that you get from a Ecartis
newlist.pl script into the Sendmail /etc/aliases file, and run the newaliases
program.  Others may prefer to create a separate aliases file for Ecartis; the
method to do this depends on how you're setting up your sendmail config file.
The direct method is to go into the /etc/sendmail.cf file, find the line
referring to /etc/aliases, and add AliasFile line below it for the Ecartis
aliases file.

Perhaps the most common pitfall of people setting up new Ecartis installations
--- or indeed, any new mail filter under Sendmail --- is the fact that many
Sendmail installations come with a program called SMRSH already enabled.
SMRSH, or the SendMail Restricted SHell, is a program that increases system
security by making sure that Sendmail only allows programs in a certain
directory to be run as mail filters.  Where this directory is varies depending
on your Sendmail installation, but a common one is /etc/smrsh.
   
If any attempt to mail Ecartis meets with a bounce message like the following,
you are using SMRSH.
   
\begin{quote}   
\footnotesize
\begin{verbatim}
----- The following addresses had permanent fatal errors -----

"|/home/ecartis/ecartis -s mylist"

     (expanded from: <mylist@mydomain.com>)


    ----- Transcript of session follows -----

sh: ecartis not available for sendmail programs

554 "|/home/ecartis/ecartis -s mylist"... Service unavailable
\end{verbatim}
\end{quote}
   
The solution is to locate the directory that SMRSH is using for its programs,
and write a shell script like this, placing it there.
   
\begin{quote}
\footnotesize
\begin{verbatim}
#!/bin/sh
/home/ecartis/ecartis $@
\end{verbatim}
\end{quote}

Of course, /home/ecartis should be the path where you installed the Ecartis
binary.  It is also vitally important to make sure that this wrapper script is
set executable so that smrsh can run it.  Also make sure that the directory
leading up to the Ecartis binary has execute permissions for whatever user or
group your mail server runs as.
   
Then you'd want to go and change the aliases so that they had the path to the
shell script wrapper.  For example, changing \ldots
   
\begin{quote}   
\footnotesize
\begin{verbatim}
mylist:           |"/home/ecartis/ecartis -s mylist"
\end{verbatim}
\end{quote} 
   
\ldots to \ldots
   
\begin{quote}
\footnotesize
\begin{verbatim}
mylist:           |"/etc/smrsh/ecartis -s mylist"
\end{verbatim}
\end{quote}
   
\ldots then Sendmail should allow Ecartis to be used as a mail filter.
   
\subsection{Exim}
\label{starting:filters:exim}

Exim is another of the mail servers out there, though it doesn't come
preinstalled on many --- if any --- systems.  Information on it can be found
at \hturl{http://www.exim.org/}.  Exim was developed at the University of
Cambridge over in England.  To make Ecartis work with Exim, you could take the
simple approach like with Sendmail and paste the Ecartis aliases into the
existing Exim aliases file, or you can go into exim.conf and add the following
lines:
   
\begin{quote}
\footnotesize
\begin{verbatim}
ecartis_aliases:
  driver = aliasfile
  file_transport = address_file
  pipe_transport = address_pipe
  file = /usr/lib/ecartis/aliases
  search_type = lsearch
  user = ecartis
  group = ecartis
\end{verbatim}
\end{quote}
   
Other than that, Ecartis should function correctly with Exim out-of-the-box.
   
\subsection{Postfix}
\label{starting:filters:postfix}
   
Postfix (\hturl{http://www.postfix.org}) is another mail server, designed with
the goal of creating a package as secure and fast as sendmail, while still
providing as much backwards compatibility as possible.  For Postfix
installations, you can once again take the simple route and paste the Ecartis
aliases into the default aliases file and then run the postaliases program any
time you change it.
   
If you wish to have a separate Ecartis aliases file, you will need to pull
/etc/postfix/main.cf up in your favorite text editor.  Find alias\_maps line,
and add the ecartis aliases file to it.  The resulting line will look
something like:
   
\begin{quote}   
\footnotesize
\begin{verbatim}
alias_maps = hash:/etc/aliases, hash:/etc/mail/ecartis.aliases
\end{verbatim}
\end{quote}
   
Then run postalias /etc/mail/ecartis.aliases every time you add aliases to the
Ecartis file.  Of course, /etc/mail/ecartis.aliases should be replaced by the
path to where you keep your Ecartis aliases file.
   
\subsection{qmail}
\label{starting:filters:qmail}

The qmail (\hturl{http://www.qmail.org}) program is perhaps the second most
widely used UNIX mail server on the Internet, being considered to be small,
fast and secure.  However, the method of setting up qmail aliases is far
different from setting up aliases under any other mail package.  Instead of a
single file that contains multiple aliases mapping through to Ecartis, you
need to create what are called dot-qmail files in the home directory of the
qmail system user (often `alias').
   
If the line in a normal aliases file would be \ldots

\begin{quote}
\footnotesize
\begin{verbatim}
mylist:           |"/home/ecartis/ecartis -s mylist"
\end{verbatim}
\end{quote}
   
\ldots then you would need to create a file in the home directory of the qmail
system user.  This file would be named .qmail-mylist and which will contain
the text \verb+|"/home/ecartis/ecartis -s mylist"+
   
You could create such a file by going to the appropriate directory and typing:
   
\begin{quote}
\footnotesize
\begin{verbatim} 
echo "|/home/ecartis/ecartis -s mylist" > .qmail-mylist
\end{verbatim}
\end{quote}

In other words, for a line like \ldots

\begin{quote}   
\footnotesize
\begin{verbatim}
address:          command
\end{verbatim}
\end{quote}   
   
\ldots you want \ldots
   
\begin{quote}
\footnotesize
\begin{verbatim}
echo "command" > .qmail-address
\end{verbatim}
\end{quote}   
   
Unlike most mail servers, you do not need to run a program to rebuild aliases
under qmail.
   
Ecartis's internal newlist command can create dot-qmail files for a given
list; to set up your Ecartis installation to do so by default, pull
ecartis.cfg up in your favorite text editor, and add the following line:

\begin{quote}
\footnotesize
\begin{verbatim}
newlist-qmail = yes
\end{verbatim}
\end{quote}
   
Then, when you create a list, it will have a subdirectory called
qmail-aliases, and in that directory will be all the dot-qmail files you need
to copy over to your qmail global aliases directory.
   
You can force this feature on by adding the command-line argument -qmail when
creating a new list.
   
It is also worth noting that there is an optional add-on package for qmail by
Dan Bernstein called FastForward
(\hturl{http://www.pobox.com/~djb/fastfoward.html}) which allows qmail to use
/etc/aliases.
   
\chapter{Configuring the Server}
\label{configuring}

\section{Primary (Global) Config File}
\label{configuring:global}

   
\chapter{Setting Up a List}
\label{newlist}

\section{Creating the List}
\label{newlist:create}

Ecartis contains a set of commands internally to make it easier to create a
list. This entire process can be invoked from the command line and easily
controlled, and the various installed modules are responsible for generating
the configuration file and documenting it, as well as any other custom
handling for new list creation.
   
To begin this process, you will want to run the Ecartis binary with the
command-line argument -newlist; this takes one parameter, the name of the new
list. In addition, if you're creating a list for a virtual host, simply put
the virtual host config file at the beginning of the line. Ecartis determines
what path to put in the aliases by what path you use when invoking it, so in
this case be sure to run it with a full path name. Here are two examples:

\begin{quote} 
\footnotesize
\begin{verbatim}
/home/ecartis/ecartis -newlist testlist
\end{verbatim}
\end{quote}

\begin{quote}
\footnotesize
\begin{verbatim}
/etc/smrsh/ecartis -c virthost1.cfg -newlist testlist2
\end{verbatim}
\end{quote}
   
Ecartis will prompt you for the e-mail address of the list administrator; this
should be an address from which mail can be sent as well as received. Do not
use a forwarding address for the primary administrator of the list. Once
you've entered this information, the script will create the basic list
configuration and basic list directory. Then it will give you a block of
information that looks something like this:
   
\begin{quote}   
\footnotesize
\begin{verbatim}
# Aliases for list 'testlist'
testlist: "|/home/ecartis/ecartis -s testlist"
testlist-request: "|/home/ecartis/ecartis -r testlist"
testlist-repost: "|/home/ecartis/ecartis -a testlist"
testlist-admins: "|/home/ecartis/ecartis -admins testlist"
testlist-moderators: "|/home/ecartis/ecartis -moderators testlist"
testlist-bounce: "|/home/ecartis/ecartis -bounce testlist"
testlist-owner: joe@mydomain.com
\end{verbatim}
\end{quote}
   
You will want to put this information into your mail server's alias file, and
you may need to run a program to rebuild the aliases database (for example,
Sendmail users will need to run the newaliases program).
   
As a side note for advanced users, all the output Ecartis gives in this mode
OTHER than the aliases is on stderr, while the aliases are printed to stdout,
thus making it easy to simply redirect output and append it to your aliases
file.
   
If you have Ecartis set up to use qmail instead (see
Section~\ref{starting:filters:qmail}), you will not get the block of aliases,
but will instead find that in the directory of your new mailing list is a
subdirectory called qmail-aliases; in this directory are the dot-qmail files
you will want to copy over to your qmail global aliases directory.
   
\section{Editing the List Configuration}
\label{newlist:configure}
   
Now, you doubtless want to go tune a few of the default settings in the list
configuration file.  Go to the directory where you have your Ecartis lists
stored, and go into the directory with the name of the list you just made.
Bring the config file up in your favorite editor, and edit to taste.  There is
an appendix of configuration variables at the end of this document.
   
   
This section needs more detail.
   
Once you have things configured to your taste, you're ready to move on to your
first interaction with Ecartis!
   
\chapter{Interacting with Ecartis}
\label{interface}
   
\section{Concepts}
\label{interface:concepts}
   
Unlike many MLMs, Ecartis has something called a `list context.'  What this
means is that, much like someone in a conversation, Ecartis remembers what the
previous command in a session referred to.  If you tell it to do a command on
list1 and then don't give it a list name for the next command, it will assume
you are still talking about list1.  If, however, you tell it to refer to list2
in the second command, it will do so.  How this works will become clearer as
we move along.
   
Ecartis also has what is called a `secure' mode.  Certain commands --- such as
administrative functions --- must be performed through a method that prevents
people from pretending to be you.  Some MLMs only check the address a message
comes from before allowing administrative commands, leaving them open to easy
hacking by a mail-spoofer.  Other MLMs use passwords, which could be
discovered.  Ecartis uses a unique method which secures it in any situation
but that where either the machine that Ecartis runs on has been compromised,
or the e-mail account of an administrator has (something out of the realm of
control of an MLM).  This method is referred to as using `cookies.'
   
Ecartis's cookies are used for more than just administrative modes, as well.
When a cookie is ``baked,'' (created by Ecartis) it has a specific ``flavor.''
A cookie of one flavor cannot be used for a task requiring a different flavor
--- for example, a cookie for a user to confirm their subscription could not
be used to authorize administrative commands.  If a cookie is used, it is
``eaten'' and cannot be used again.  If a cookie remains unused, eventually it
goes stale and is removed from the cookie jar.  And lastly, a cookie can be
baked for a specific person, and then only that person can use it.
   
It is the cookie system that allows Ecartis to have much of the security and
power it does.
   
\section{Normal Mode}
\label{interface:normal}

Normal mode is how the majority of users interact with Ecartis.  This is where
you send e-mail to the Ecartis server itself with a set of commands.  For
example, a user might send a message like this \ldots
   
\begin{quote}   
\footnotesize
\begin{verbatim}
Date: Tue, 28 Sep 1999 23:26:32 -0700 (PDT)
From: Joe Q. User <joe@someisp.com>
To: ecartis@mydomain.com
Subject:

lists
which
end
\end{verbatim}
\end{quote}
   
And Ecartis would send back a message like \ldots
   
   
\begin{quote}
\footnotesize
\begin{verbatim}
Date: Tue, 28 Sep 1999 23:27:27 -0700 (PDT)
From: Ecartis <ecartis@mydomain.com>
To: joe@someisp.com
Subject: Ecartis command results: lists

>> lists
Ecartis lists available on this machine:

testlist
          A test list.

>> which
Retrieving list subscriptions.

joe@someisp.com is subscribed to the following lists:
      testlist

>> end
Command set concluded.  No further commands will be processed.

---
Ecartis v0.126a - job execution complete.
\end{verbatim}
\end{quote}   
   
In other words, each line of the message is parsed by Ecartis, which tries to
find a valid command in it.  If there is a valid command in the subject line,
it will use that instead of parsing the body (unless configured to ignore the
subject line).  What commands are available vary depending on the Ecartis
installation, since an LPM can add new commands.
   
Additionally, if you enclose what is called a ``job/eoj wrapper,'' Ecartis
will ignore everything except what is between the beginning and ending of the
wrapper.  (Multiple job/eoj wrappers can be used per message.)  This is useful
for administrative functions, as you will see later.
   
\begin{quote}
\footnotesize
\begin{verbatim}
// job
commands
// eoj
\end{verbatim}
\end{quote}
   
   
When a command takes a list as one of the parameters, it turns that into the
list `context,' and if you omit a list from the next command, it will assume
you mean the same list as before.  When a command finally has another list
given, it will change the context for following commands.  Whenever a command
changes the list context, you will be told in the results list:
   
\begin{quote}
\footnotesize
\begin{verbatim}
List context changed to 'testlist' by following command.
>> who testlist
Membership of list 'testlist':
        joe@mydomain.com (ADMIN)
        jane@mydomain.com (ADMIN)
        joe@someisp.com

>> stats
Current account flags for 'joe@mydomain.com' on 'testlist':

        REPORTS
        CCERRORS
        ECHOPOST
        MODERATOR

List context changed to 'testlist2' by following command.
>> stats testlist2
Current account flags for 'joe@mydomain.com' on 'testlist2':

        ADMIN
        SUPERADMIN
        ECHOPOST
        REPORTS
        CCERRORS

>> who
Membership of list 'testlist2:
        joe@mydomain.com (ADMIN)
        joe@someisp.com
\end{verbatim}
\end{quote}
   
In the above example, the user (joe@mydomain.com) sent the commands:
   
   
\begin{quote}
\footnotesize
\begin{verbatim}
who testlist
stats
stats testlist2
who
\end{verbatim}
\end{quote}
   
Now you know how to issue commands to Ecartis, but what commands can you
issue?  Well, there are a great many of them, and they vary from installation
to installation because of customization.  Luckily, there is a way to query
Ecartis for what commands it supports.  If you send Ecartis the command
commands, it will send back a list of what commands it supports on a given
installation.  Some commands are only ever useful in messages Ecartis
generates, such as for secured-mode messages.
   

\section{Secure (Administrator) Mode}
\label{interface:admin}

Obviously, there needs to be a way to issue administrative commands to
Ecartis; it isn't feasible to expect every user to have access to the files on
disk that control a list they might potentially be an administrator on.  But
it is equally obvious that it's undesirable to have random unauthorized users
able to issue administrative commands.  Clearly, some sort of authentication
is needed.  Ecartis achieves this authentication with cookies, as mentioned
before.
   
There are two ways to handle administrative commands, but they have the same
end result.  The first method is to send Ecartis the command admin list.  If
you are a valid administrator for list, Ecartis will send you what is called
an admin wrapper.  This is something that looks like:
   
\begin{quote}
\footnotesize
\begin{verbatim}
// job
adminvfy testlist 37F1C6FC:58BB.1:ybxvznvfbabgnxharg
adminend
// eoj
\end{verbatim}
\end{quote}
   
You want to fill out any administrative commands (things like getconf, or
setfor, which require administrative permissions) between the adminvfy and
adminend lines, and send the wrapper back to Ecartis.
   
The second method was added in a later version of Ecartis, and is more
convenient in most cases.  Instead of sending the command admin list, send the
command admin2 list, and then give it the commands you want in the wrapper
followed by adminend2.  That will send back the wrapper already filled out,
and you can simply hit `reply' to approve it.  This is also where job/eoj
blocks come in handy!  Picture receiving a note from a user (say,
joe@someisp.com) which says that they need to be unsubscribed and can't
remember how.  You could simply reply and carbon-copy Ecartis, saying
something like:
   
\begin{quote}   
\footnotesize
\begin{verbatim}
> Hey, sorry... I know I should know how to unsubscribe myself
> But...
> Could you unsubscribe me for me?

Sure!

// job
admin2 testlist
unsubscribe joe@someisp.com
adminend2
// eoj
\end{verbatim} 
\end{quote}
   
Then you would receive back an already filled-out administrative wrapper such
as:
   
   
\begin{quote}
\footnotesize
\begin{verbatim}
// job
adminvfy testlist 37F1C6FC:58BB.1:ybxvznvfbabgnxharg
unsubscribe joe@someisp.com
adminend
// eoj
\end{verbatim}
\end{quote}
   
   
Ecartis can parse reply-formatted messages in many cases (the exceptions being
the putconf command and moderated messages), so when you use the admin2
command, you can simply hit reply to the wrapper it generates.
   
Any commands that are listed with an (ADMIN) after them in the commands list
that Ecartis will generate will only work between an adminvfy line and an
adminend line.  Normal commands will also work between those lines, but some
have different uses; for example, the stats command normally functions as
stats list, to allow a user to view statistics on themselves for a list.  But
in admin mode, you are locked to a specific list - the list you validated
administrator permissions for - so the command you would enter instead is
stats user, to allow an administrator to view stats for a specific user on
that list.
   
\section{Web Interface}
\label{interface:lsg2}

Information on LSG/2 needs to go here (after LSG/2 is finished).
   
\chapter{List Moderation}
\label{moderation}

Information on moderating mailing lists.
   
\chapter{Modules}

   
Will give a listing of the default Ecartis modules (administrivia, etc.) and
what they provide.
   
   
\chapter{Example List Configurations}

Some useful list configurations, like an announcement list or a connected set
of lists like the nwcpp-discuss and nwcpp-announce lists, or the
ecartis-support/ecartis-dev/ecartis-announce setup.
   
   
\chapter{Writing Modules}

Instructions on how to write a Ecartis module.


% \backmatter


\appendix
\chapter{Config Variables}
\label{config}

\textbf{Ecartis 1.0.2 Variable Reference}


Note: The 'valid' field describes the Ecartis config files where that variable is valid.
'G' means the global config file, 'V' means a virtual host configuration file, and 'L'
means individual list files.

ACCOUNT MANAGEMENT

\begin{longtable}{p{30mm}llp{60mm}}
\hline
Variable & Type & Valid & Description\\
\hline
allow-setaddy & boolean & GVL & Allow the use of the setaddy command to replace the subscribed
address. Example: allow-setaddy=no

Default value is true\\
\hline
\multicolumn{4}{l}{ADDRESS HANDLING} \\

\hline
Variable & Type & Valid & Description\\
\hline
deny-822-bounce & boolean & GVL & Should the RFC822 Resent-From: header be trusted for sender.
Example: deny-822-bounce = yes
Default value is false\\
\hline
deny-822-from & boolean & GVL & Should the RFC822 From: header be trusted for sender.
Example: deny-822-from = no

Default value is false \\
\hline
\multicolumn{4}{l}{ADDRESSES}\\
\hline
Variable & Type & Valid & Description\\
\hline
administrivia-address & string & GVL & Address to which subscription/unsubscription attempt notifications should be sent.
Example: administrivia-address = mylist-admins@host.dom\\
\hline
approved-address & string & GVL & Address to which approved/rejected/modified moderated posts should be sent.
Example: approved-address = mylist-repost@myhost.dom Default value is \textless\$list\textgreater-repost@\textless\$hostname\textgreater\\
\hline
listserver-address & string & GV & The email address for the listserver control account.
Example: listserver-address = listserv@myhost.dom

Default value is ecartis@localhost\\
\hline
listserver-admin & string & GV & The email address of the human in charge of the listserver.
Example: listserver-admin = user1@host2.dom

Default value is root@localhost\\
\hline
listserver-full-name & string & GV & The friendly name used to identify the listserver.
Example: listserver-full-name = List Server

Default value is Ecartis \\
\hline
\multicolumn{4}{l}{ADMINISTRATION}\\
\hline
Variable & Type & Valid & Description\\
\hline
admin-actions-shown & boolean & GVL & Are administrator actions shown to list subscribers (i.e. subscribe/unsubscribe
show the administrator email address if actions are shown) Example: admin-actions-shown = no Default value is true\\
\hline
paranoia & boolean & GVL & Are the various list config files allowed to be edited remotely for this list.
Example: paranoia = yes

Default value is false\\
\hline
\multicolumn{4}{l}{ADMINISTRIVIA}\\
\hline
Variable & Type & Valid & Description\\
\hline
administrivia-body-lines & boolean & GVL & Are administrator actions shown to list subscribers (i.e. subscribe/unsubscribe
show the administrator email address if actions are shown) Example: admin-actions-shown = no

Default value is true\\
\hline
administrivia-check & boolean & GVL & Are the various list config files allowed to be edited remotely for this list.
Example: paranoia = yes

Default value is false\\
\hline
administrivia-regexp-file & boolean & GVL & Are the various list config files allowed to be edited remotely for this list.
Example: paranoia = yes

Default value is false\\
\hline
\multicolumn{4}{l}{ANTISPAM}\\
\hline
Variable & Type & Valid & Description\\
\hline
allow-spam & boolean & GVL & Should we disable the antispam check for this list.
Example: allow-spam = false

Default value is false\\
\hline
antispam-blackhole & boolean & GVL & If we receive spam, should we simply eat it? (If 'no', then it is moderated.)
Example: antispam-blackhole = yes

Default value is false\\
\hline
spamfile & string & GVL & The file on disk which contains the regular expressions used to detect if a given sender is a spammer.
Example: spamfile = spam-regexp

Default value is spam-regexp\\

\hline
\multicolumn{4}{l}{BASIC CONFIGURATION}\\
\hline
Variable & Type & Valid & Description\\
\hline
advertise & boolean & GVL & Does this list show up as being available.
Example: advertise = false

Default value is true\\
\hline
config-file & string & GV & The name of the list-specific configuration file.
Example: config-file = config

Default value is config\\
\hline
default-flags & string & GVL &Default flags given to a user when they are subscribed.
Example: default-flags = $\mid$NOPOST$\mid$DIGEST$\mid$

Default value is $\mid$ECHOPOST$\mid$\\
\hline
description & string & GVL & Description of the list.
Example: description = This is my special list\\
\hline
hostname & string & GV & Hostname for URLs/addresses/headers
Example: hostname = lists.mydomain.com\\
\hline
ignore-subject-commands & boolean & GV & Should the server ignore commands in the subject line.
Example: ignore-subject-commands = false

Default value is false\\
\hline
list-owner & string & GVL & Defines an email address to reach the list owner(s).
Example: list-owner = list2-admins@hostname.dom\\
\hline
reply-to & string & GVL & Address which will appear in the Reply-To: header.
Example: reply-to = list@myhost.dom\\
\hline
subject-tag & string & GVL & Optional tag to be included in subject lines of posts sent to the list.
Example: subject-tag = [MyList]\\
\hline
task-no-footer & boolean & GVL & Should the messages produced by the server have a footer with version information
Example: task-no-footer = yes

Default value is false\\
\hline
\multicolumn{4}{l}{BOUNCE HANDLING}\\
\hline
Variable & Type & Valid & Description\\
\hline
bounce-always-unsub & boolean & GVL & Should the user be unsubscribed when more than max transient bounces have occured.
Example: bounce-always-unsub = false

Default value is false\\
\hline
bounce-max-fatal & integer & GVL & Maximum number of fatal bounces before action is taken.
Example: bounce-max-fatal = 10

Default value is 10\\
\hline
bounce-max-transient & integer & GVL & Maximum number of transient bounces before action is taken.
Example: bounce-max-transient = 100

Default value is 30\\
\hline
bounce-never-unsub & boolean & GVL & Should the user be unsubscribed when more than max fatal bounces have occured, or just set vacation.
Example: bounce-never-unsub = off

Default value is false\\
\hline
bounce-never-vacation & boolean & GVL & Should the user ever be set vacation for exceeding the maximum number of bounces.
Example: bounce-never-vacation = yes

Default value is false\\
\hline
bounce-timeout-days & integer & GVL & Length of time (in days) during which the maximum number of bounces must not be exceeded.
Example: bounce-timeout-days = 7

Default value is 7\\
\hline
bouncer-unsub-file & string & GVL & File under the list directory to send to a user when they are automatically unsubscribed by the bouncer.
Example:  bouncer-unsub-file = text/bounce-unsub.txt

Default value is text/bounce-unsub.txt\\
\hline
bouncer-vacation-file & string & GVL & File under the list directory to send to a user when they are automatically set vacation by the bouncer.
Example:  bouncer-vacation-file = text/bounce-vacation.txt

Default value is text/bounce-vacation.txt\\
\hline
\multicolumn{4}{l}{CGI}\\
\hline
Variable & Type & Valid & Description\\
\hline
cgi-template-dir & string & GVL & Directory for CGI gateway templates.
Example:  cgi-template-dir = \textless\$listserver-data\textgreater/templates

Default value is \textless\$listserver-data\textgreater/templates\\
\hline
lsg2-cgi-url & string & GV & URL on the associated web server pointing to the LSG/2 cgi wrapper.
Example:  lsg2-cgi-url = http://my.dom/cgi-bin/lsg2.cgi\\
\hline
lsg2-cookie-duration & duration & GV & The length of time that cookies for the LSG/2 web form should last.
Example: lsg2-cookie-duration = 30 m

Default value is 15 m\\
\hline
lsg2-iis-support & boolean & G & Does LSG/2 need to run in Microsoft IIS-compatible mode? (Likely breaks other webservers.)
Example:  lg2-iis-support = yes

Default value is false\\
\hline
lsg2-paranoia & boolean & GV & Is LSG/2 paranoid, e.g. does it deny all remote administration.
Example: lsg2-paranoia = no

Default value is false\\
\hline
lsg2-sort-userlist & boolean & GVL & Should LSG/2 sort the userlist in admin mode. This can be expensive for large lists.
Example:  lsg2-sort-userlist = true

Default value is true\\
\hline
\multicolumn{4}{l}{COOKIES}\\
\hline
Variable & Type & Valid & Description\\
\hline
expire-all-cookies & boolean & GV & Should we expire cookies for all lists on initial run? Should only be set to 'no'
on installations with a huge (multi-thousand) number of lists.
Example: expire-all-cookies = yes

Default value is true\\
\hline
\multicolumn{4}{l}{DEBUGGING}\\
\hline
Variable & Type & Valid & Description\\
\hline
debug & integer & GVL & How much logging should be done.
Example: debug = 10

Default value is 0\\
\hline
logfile & integer & GV & Filename where debugging log information will be stored.
Example:  logfile = ./server.log\\
\hline
preserve-queue & boolean & GVL & Controls whether to remove queue file after processing.
Example:  preserve-queue = yes

Default value is false\\
\hline
validate-users & boolean & GVL & Perform a minimal validation of user@host.dom on all users
in the list's user file and log errors. Example:   validate-users = true

Default value is false\\
\hline
\multicolumn{4}{l}{DIGEST}\\
\hline
Variable & Type & Valid & Description\\
\hline
digest-administrivia-file & string & GVL & File on disk used to store digest administrative information.
Example:  digest-administrivia-file = digest/administrivia

Default value is digest/administrivia\\
\hline
digest-alter-datestamp & boolean & GVL & Should digests use a different datestamp format.
Example:  digest-alter-datestamp = on

Default value is false\\
\hline
digest-altertoc & boolean & GVL & Should this list use an alternate form for the digest Table of Contents.
Example:  digest-altertoc = false

Default value is false\\
\hline
digest-footer-file & string & GVL & Filename for a footer file automatically included with every digest
Example:  digest-footer-file = text/digest-footer.txt

Default value is text/digest-footer.txt\\
\hline
digest-from & string & GVL & Email address used as the From: header when the digest is distributed.
Example: digest-from = listname@host.dom\\
\hline
digest-header-file & string & GVL & Filename for a header file automatically included with every digest
Example:  digest-header-file = text/digest-header.txt

Default value is text/digest-header.txt\\
\hline
digest-max-size & integer & GVL & Maximum size a digest can reach before being automatically sent.
Example:  digest-max-size = 40000

Default value is 0\\
\hline
digest-max-time & duration & GVL & Maximum age of a digest before it is sent automatically.
Example: digest-max-time = 24h

Default value is 0s\\
\hline
digest-name & string & GVL & If digests are kept, what do we use as the name template for the stored copy of the digest.
Example:
  digest-name = digests/\%l/V\%V.I\%i\\
\hline
digest-no-fork & boolean & GVL & Should digesting be done by forking a seperate process.
Example:  digest-no-fork = true

Default value is false\\
\hline
digest-no-toc & boolean & GVL &S Should digests exclude the Table of Contents entirely.
Example:  digest-no-toc = TRUE

Default value is false\\
\hline
digest-no-unmime & boolean & GVL & Should posts in the digest not be unmimed.
Example:  digest-no-unmime = off

Default value is false\\
\hline
digest-send-mode & choice & GVL & Mode used when sending digests daily via a timed job (usually around midnight
of the host machine's time). 'procdigest' means that when that happens, the digest will be sent regardless of
your time and size settings (which are still honored for normal posts). Time and size are self-explanatory;
time means that it will only send if there's been more than digest-max-time elapsed, while size will only
send if digest-max-size has been exceeded. digest-max-size and digest-max-time DO still apply when individual
posts come across the list, even if procdigest is set; having digest-max-size set to 50000 and this variable
to procdigest would mean that the digest would be sent when it exceeded 50k, or during the midnight automated
run (perhaps the day's digest only reached 20k; it would still be sent).
Example:  digest-send-mode = time

Default value is procdigest\\
\hline
digest-strip-tags & boolean & GVL & Should subject lines of the messages in the digest have the list subject-tag stripped.
Example:  digest-strip-tags = on

Default value is false\\
\hline
digest-to & string & GVL & Email addres used as the To: header when the digest is distributed.
Example:  digest-to = listname@host.dom\\
\hline
digest-transient & boolean & GVL & Are digests removed completely after they are sent.
Example:  digest-transient = off

Default value is true\\
\hline
digest-transient-administrivia & boolean & GVL & Should the digest administrivia file be removed after the digest is next sent.
Example:  digest-transient-administrivia = true

Default value is false\\
\hline
no-digest & boolean & GVL & Should digesting be disabled for this list.
Example:  no-digest = yes

Default value is false\\
\hline
\multicolumn{4}{l}{DUPLICATE MESSAGE DETECTION}\\
\hline
Variable & Type & Valid & Description\\
\hline
no-dupes & boolean & GVL & Should we track the Message-Id headers from incoming traffic to prevent duplicate posts to the list. Message-Id's expire after 1 day.
Example:  no-dupes = off

Default value is true\\
\hline
no-dupes-forever & boolean & GVL & Should we never expire the Message-Id's.
Example:  no-dupes-forever = yes

Default value is false\\
\hline
\multicolumn{4}{l}{ERROR HANDLING}\\
\hline
Variable & Type & Valid & Description\\
\hline
error-include-queue & boolean & GVL & Should error reports contain the queue associated with that run
Example:  error-include-queue = yes

Default value is true\\
\hline
\multicolumn{4}{l}{FILEARCHIVE}\\
\hline
Variable & Type & Valid & Description\\
\hline
file-archive-dir & string & GVL & Where are the archives for the list located.
Example: file-archive-dir = files\\
\hline
file-archive-status & choice & GVL & Who is allowed to retrieve the file archives.
Example: file-archive-status = admin

Default value is public\\
\hline
\multicolumn{4}{l}{FILES}\\
\hline
Variable & Type & Valid & Description\\
\hline
blacklist-mask & string & GVL & Per-list file containing regular expressions for users who are not allowed to subscribe to the list.
Example:  blacklist-mask = blacklist

Default value is blacklist\\
\hline
blacklist-reject-file & string & GVL & File sent to a user when their subscription or post is rejected because they are blacklisted.
Example:   blacklist-reject-file = text/blacklist.txt

Default value is text/blacklist.txt\\
\hline
closed-file & string & GVL & File sent to message author when the post is rejected because the list is closed.
Example:  closed-file = text/closed-post.txt

Default value is text/closed-post.txt\\
\hline
closed-subscribe-file & string & GVL & Filename of file to be sent if a user tries to subscribe to a closed subscription list.
Example:  closed-subscribe-file = text/closed-subscribe.txt

Default value is text/closed-subscribe.txt\\
\hline
faq-file & string & GVL & File on disk containing the list's FAQ file.
Example:  faq-file = text/faq.txt

Default value is text/faq.txt\\
\hline
footer-file & string & GVL & Text to append to list messages.
Example:  footer-file = text/footer.txt

Default value is text/footer.txt\\
\hline
global-blacklist & string & GV & Global file containing regular expressions for users who are not allowed to subscribe to lists hosted on this server.
Example:  global-blacklist = banned

Default value is banned\\
\hline
goodbye-file & string & GVL & File sent to someone unsubscribing from a list.
Example:  goodbye-file = text/goodbye.txt

Default value is text/goodbye.txt\\
\hline
header-file & string & GVL & Text to prepend to list messages.
Example:  header-file = text/header.txt

Default value is text/header.txt\\
\hline
info-file & string & GVL & File on disk containing the list's info file.
Example:  info-file = text/info.txt

Default value is text/info.txt\\
\hline
moderator-welcome-file & string & GVL & File sent to a new moderator when they set MODERATOR.
Example:  moderator-welcome-file = text/moderator.txt

Default value is text/moderator.txt\\
\hline
no-command-file & string & GV & This is a global file to send if a message to the main listserver or request address has no commands.
Example:  no-command-file = helpfile

Default value is ecartis.hlp\\
\hline
nopost-file & string & GVL & File sent to users flagged NOPOST if they submit a post to the list
Example:  nopost-file = text/nopost.txt

Default value is text/nopost.txt\\
\hline
outside-file & string & GVL & File sent to message author when post is accepted to list but author is not a subscriber.
Example:  outside-file = text/outside.txt

Default value is text/outside.txt\\
\hline
overquote-file & string & GVL & Filename under the list directory of the file sent to a user who fails an overquoting check. (See 'quoting-limit'.) This is the file retrieved with 'getconf overquote'.
Example:  overquote-file = text/overquote.txt

Default value is text/overquote.txt\\
\hline
per-user-footer-file & string & GVL & Filename of the per-user footer file.
Example:  per-user-footer-file = per-user/footer.txt

Default value is text/peruser-footer.txt\\
\hline
posting-acl-file & string & GVL & The filename within the list directory to use for the posting-acl file.
Example:  posting-acl-file = postacl

Default value is postacl\\
\hline
submodes-file & string & GVL & File containing list specific customized subscription modes.
Example:  submodes-file = submodes

Default value is submodes\\
\hline
subscribe-acl-file & string & GVL & File containing regular expressions against which a user's address will be matched when they try to subscribe to a list. If an address does not match at least one, subscription is denied. Can be gotten with 'getconf acl'.
Example:  subscribe-acl-file = subscribe-acl

Default value is subscribe-acl\\
\hline
subscribe-acl-text-file & string & GVL & Textfile to be sent to a user who fails the ACL subscription check. Can be gotten with 'getconf acl-text'.
Example:  subscribe-acl-text-file = text/subscribe-acl-deny.txt

Default value is text/subscribe-acl-deny.txt\\
\hline
tempban-end-file & string & GVL & Filename of file to be sent to a user who was tempbanned when the tempban expires.
Example:  tempban-end-file = text/tempban-end.txt

Default value is text/tempban-end.txt\\
\hline
tempban-file & string & GVL & Filename of file to be sent to a user when an admin issues the tempban command on them.
Example:  tempban-file = text/tempban.txt

Default value is text/tempban.txt\\
\hline
welcome-file & string & GVL & File sent to new subscribers of a list.
Example:  welcome-file = text/intro.txt

Default value is text/intro.txt\\
\hline
\multicolumn{4}{l}{HEADERS}\\
\hline
Variable & Type & Valid & Description\\
\hline
strip-headers & string & GVL & A colon seperated list of headers to remove from outgoing messages.
Example:  strip-headers = X-pmrq:X-Reciept-To\\
\hline
strip-mdn & boolean & GVL & If true, strip all read-reciept (mail delivery notification) headers
from mail before sending out. Example:  strip-mdn = on

Default value is true\\
\hline
\multicolumn{4}{l}{LIST INTEGRATION}\\
\hline
Variable & Type & Valid & Description\\
\hline
cc-lists & string & L & A colon seperated list of local lists which recieve copies of all posts to this list.
Example:  cc-lists = mylist1:mylist2

Default value is \\
\hline
union-lists & boolean & L & A colon seperated list of local lists whose members can post to this list even if it is closed.
Example:  union-lists = mylist1:mylist2\\
\hline
\multicolumn{4}{l}{LISTARCHIVE}\\
\hline
Variable & Type & Valid & Description\\
\hline
archive-world-readable & boolean & GVL & Should we make all archive files world-readable?
Example:  archive-world-readable = yes

Default value is true\\
\hline
mbox-archive-path & string & GVL & Path to where MBox format archives are stored.
Example:  mbox-archive-path = archives/mylist/mbox\\
\hline
mh-archive-path & string & GVL & Path to where MH format archives are stored.
Example:  mh-archive-path = archives/mylist/mh\\
\hline
\multicolumn{4}{l}{LOCATION}\\
\hline
Variable & Type & Valid & Description\\
\hline
lists-root & string & GV & Location of the directory containing all the list info.
Example:  lists-root = lists

Default value is \textless\$listserver-data\textgreater/lists\\
\hline
listserver-conf & string & G & The path to the listserver configuration files.
Example: listserver-conf = /usr/local/mylists/configs\\
\hline
listserver-data & string & GV & The path to the listserver data root.
Example: listserver-data = /usr/local/mylists/data\\
\hline
listserver-modules & string & G & The path to the directory containing the LPM modules.
Example:  listserver-modules = /usr/local/lists/modules\\
\hline
listserver-root & string & G & The path to the root of the Listserver installation
Example: listserver-root = /usr/local/listserver\\
\hline
\multicolumn{4}{l}{MAINTENANCE}\\
\hline
Variable & Type & Valid & Description\\
\hline
listserver-bin-dir & string & G & When creating a new list, what directory do we prepend to
the binary name when we make the aliases (if not set, defaults to the path the binary was run with).
Example: listserver-bin-dir = /home/list\\
\hline
newlist-qmail & boolean & G & When creating a new list, do we need to make dot-qmail aliases?
Example: newlist-qmail = no

Default value is false\\
\hline
\multicolumn{4}{l}{MIME}\\
\hline
Variable & Type & Valid & Description\\
\hline
humanize-html & boolean & GVL & Should HTML attachments be converted to plaintext
Example:  humanize-html = no

Default value is true\\
\hline
humanize-mime & boolean & GVL & Should the server strip out non-text MIME attachments.
Example: humanize-mime = true

Default value is true\\
\hline
humanize-quotedprintable & boolean & GVL & If set to true, attempt to remove any and all quoted
printable characters from subject and body replacing them with their actual character. Example: humanize-quotedprintable = true

Default value is false\\
\hline
pantomime-dir & string & GVL & Directory on disk to store binary files placed on the web via PantoMIME.
Example:  pantomime-dir = /var/www/ecartis/html/pantomime\\
\hline
pantomime-url & string & GVL & URL corresponding to pantomime-dir
Example:  pantomime-url = http://www.ecartis.net/pantomime\\
\hline
rabid-mime & boolean & GVL & Should ABSOLUTELY no attachments, EVEN text/plain, be allowed
Example:  rabid-mime = no

Default value is false\\
\hline
unmime-forceweb & boolean & GVL & Should all attachments (even text/plain) be forced to the
web (pantomime-dir and pantomime-url must be set or all will be eaten)
Example:  unmime-forceweb = yes

Default value is false\\
\hline
unmime-quiet & boolean & GVL & Should the listserver report when it strips MIME attachments.
Example: unmime-quiet = no

Default value is false\\
\hline
\multicolumn{4}{l}{MISC}\\
\hline
Variable & Type & Valid & Description\\
\hline
assume-lists-valid & boolean & GV & Should we assume that all list directories are valid or should we perform checks
Example: assume-lists-valid = yes

Default value is false\\
\hline
closed-post & boolean & GVL & Is this list closed to posting from non-members.
Example: closed-post = true

Default value is false\\
\hline
closed-post-blackhole & boolean & GVL & Do messages submitted to a closed-post list by non-members get thrown away.
Example: closed-post-blackhole = yes

Default value is false\\
\hline
closed-post-subject & string & GVL & A customized subject for closed-post rejection mails.
Example: closed-post-subject = The List Is Closed\\
\hline
copy-requests-to & string & GVL & If set, all user request results will be sent to this address. Useful for debugging.
Example:
  copy-requests-to = \textless\$list\textgreater-admins@\textless\$hostname\textgreater\\
\hline
enforced-address-blackhole & boolean & GVL & If this is true and enforced-addressing-to is enabled,
posts that fail the check will be simply eaten. Otherwise, they will be marked for moderation.
Example: enforced-address-blackhole = true

Default value is false\\
\hline
enforced-addressing-to & string & GVL & If this is set, this address must be in the To or Cc
field of a post to the list. Usually, you would set the list address here.
Example: enforced-addressing-to = mylist@foo.bar.com\\
\hline
force-from-address & string & GVL & If specified, this will be used as the RFC 822 From: address.
Example:  force-from-address = list-admins@myhost.dom\\
\hline
form-show-listname & boolean & GVL & Should we use the list name (or RFC2369 name) instead
of the listserver full name for forms on a per-list basis? (Like admin wrappers and such.)
Example:  form-show-listname = yes

Default value is false\\
\hline
posting-acl & boolean & GVL & Should we check the postacl file if it exists?
Example:  posting-acl = true

Default value is true\\
\hline
precedence & string & GVL & The precedence header which will be included in all traffic to the list.
Example: precedence = bulk

Default value is bulk\\
\hline
reply-to-sender & boolean & GVL & Forcibly set the Reply-To: header to be the address the mail came from.
Example: reply-to-sender = false

Default value is false\\
\hline
tag-to-front & boolean & GVL & If set to true and there is a subject tag for the list, it will be
removed and moved to the beginning of the line (before any 'Re:'s). If not set, any duplicate Re:'s will
still be removed, and the subject-tag will be moved after the Re:.
Example:  tag-to-front = no

Default value is true\\
\hline
who-status & choice & GVL & Who is allowed to view the list membership.
Example:  who-status = admin

Default value is private\\
\hline
\multicolumn{4}{l}{MODERATION}\\
\hline
Variable & Type & Valid & Description\\
\hline
admin-approvepost & boolean & GVL & Are posts by an administrator to a moderated list automatically approved.
Example: admin-approvepost = false

Default value is true\\
\hline
moderate-include-queue & boolean & GVL & Should moderated messages contain the full message that triggered moderation?
Example: moderate-include-queue = yes

Default value is false\\
\hline
moderate-notify-nonsub & boolean & GVL & Should posts from non-subscribers be acked if they are moderated.
Example: moderate-notify-nonsub = true

Default value is false\\
\hline
moderated & boolean & GVL & Is this list moderated.
Example:  moderated = yes

Default value is false\\
\hline
moderator & string & GVL & Address for the list moderator(s).
Example:  moderator = foolist-moderators@hostname.dom\\
\hline
moderator-approvepost & boolean & GVL & Are posts by a moderator to a moderated list automatically approved.
Example: moderator-approvepost = false

Default value is true\\
\hline
password-failure-blackhole & boolean & GVL & If true, a post to a password list that doesn't have
the correct password will be eaten. If false, it will be sent to the moderators.
Example:  password-failure-blackhole = yes

Default value is true\\
\hline
password-implies-approved & boolean & GVL & If true, a correct use of X-posting-pass automatically
preapproves the message. Useful if you want to have a moderated list and allow preapproved functionality through the use of the password.
Example:  password-implies-approved = yes

Default value is false\\
\hline
post-password & string & GVL & If specified, all incoming messages must have an X-posting-pass: header with this password in it.
Example:  post-password = NeverBeGuessed\\
\hline
post-password-reject-file & string & GVL & File sent to submitters to a password protected list if they don't include the posting password header.
Example:  post-password-reject-file = text/postpassword.txt

Default value is text/postpassword.txt\\
\hline
subject-required & boolean & GVL & If set to true, then any post sent to the list without a subject will be made moderated.
Example: subject-required = yes

Default value is false\\
\hline
verbose-moderate-fail & boolean & GVL & When a moderator approves a message but it is rejected, should the message
in question be included in the rejection note?
Example: verbose-moderate-fail = yes

Default value is true\\
\hline
\multicolumn{4}{l}{PASSWORD}\\
\hline
Variable & Type & Valid & Description\\
\hline
allow-site-passwords & boolean & GV & Are sitewide passwords allowed.
Example: allow=site-passwords = false

Default value is false\\
\hline
password-expiration-time & duration & GV & How quickly to auth-password cookies expire.
Example:  password-expiration-time = 2 d

Default value is 1 h\\
\hline
\multicolumn{4}{l}{PER-USER}\\
\hline
Variable & Type & Valid & Description\\
\hline
per-user-rewrite-to & boolean & GVL & Should the To: field be rewritten for each user on the list?
Example:  per-user-rewrite-to = true

Default value is false\\
\hline
\multicolumn{4}{l}{POSTING LIMITS}\\
\hline
Variable & Type & Valid & Description\\
\hline
body-max-size & integer & GVL & Posts with a body larger than this in bytes will be moderated.
Example: body-max-size = 10000\\
\hline
header-max-size & integer & GVL & Posts with headers larger than this in bytes will be moderated.
Example: header-max-size = 2000\\
\hline
\multicolumn{4}{l}{QUOTING}\\
\hline
Variable & Type & Valid & Description\\
\hline
quoting-limits & boolen & GVL & If true, posts to the list will be checked for overquoting.
Example: quoting-limits = yes

Default value is false\\
\hline
quoting-line-reset & boolean & GVL & If quoting-limits is on, should the count of quoted lines be reset if the user places their own text in the message? (This has the effect of making it so that quoting-max-lines is the total number of lines that can be quoted at once IN A BLOCK, instead of in the entire message.)
Example: quoting-line-reset = yes

Default value is true\\
\hline
quoting-max-lines & integer & GVL & If greater than 0 and quoting-limits is true, this is the maximum number of lines allowed to be quoted from a previous message.
Example: quoting-max-lines = 10

Default value is 10\\
\hline
quoting-max-percent & integer & GVL & If greater than 0 and quoting-limits is true, this is the maximum percent of the message allowed to be quoted from a previous post.  Example: quoting-max-percent = 15

Default value is 0\\
quoting-tolerance-lines & integer & GVL & If greater than 0, this is the number of lines that must be exceeded in the total message before the quoting percentage limit will be applied. It would be silly to have a 25\% quoting limit and have a three-line message be rejected because two lines were quoted.
Example: quoting-tolerance-lines = 7

Default value is 7\\
\hline
\multicolumn{4}{l}{RFC2369}\\
\hline
Variable & Type & Valid & Description\\
\hline
rfc2369-archive-url & string & GVL & The URL for use in the List-archive: RFC 2369 header.
Example: rfc2369-archive-url = ftp://ftp.myhost.dom/lists\\
\hline
rfc2369-headers & boolean & GVL & Should RFC 2369 headers be enabled for this list.
Example: rfc2369-headers = on

Default value is false\\
\hline
rfc2369-legacy-listid & boolean & GVL & Add older X-List-ID headers, for backwards compatibility.
Example: rfc2369-legacy-listid = false

Default value is true\\
\hline
rfc2369-list-help & string & GVL & URL to use in the RFC 2369 List-help: header.
Example: rfc2369-list-help = http://www.mydom.com/list/help.html\\
\hline
rfc2369-listname & string & GVL & The name of the list to use in the RFC 2369 List-name: header.
Example: rfc2369-listname = Mylist\\
\hline
rfc2369-minimal & boolean & GVL & Should only the minimal set of RFC 2369 headers be emitted.
Example: rfc2369-minimal = true

Default value is false\\
\hline
rfc2369-post-address & string & GVL & The mailto to be used in the RFC 2369 List-post: header.
The special value of 'closed' will show the list as closed in that header.
Example: rfc2369-post-address = myname@myhost.dom\\
\hline
rfc2369-subscribe & string & GVL & f specified, this overrides the default generated RFC2369 List-subscribe value.
Example: rfc2369-subscribe = mailto:ecartis@ecartis.net?subject=subscribe\%20ecartis-support\\
\hline
rfc2369-unsubscribe & string & GVL & If specified, this overrides the default generated RFC2369 List-unsubscribe value.
Example: rfc2369-unsubscribe = mailto:ecartis@ecartis.net?subject=unsubscribe\%20ecartis-support\\
\hline
use-rfc2919-listid-subdomain & string & GVL & Instead of the 'pure' RFC2369 List-ID field, use the suggested
'list-id' subdomain addition from RFC2919 (the successor to RFC2369)
Example:  use-rfc2919-listid-subdomain = true

Default value is false\\
\hline
\multicolumn{4}{l}{SMTP}\\
\hline
Variable & Type & Valid & Description\\
\hline
form-cc-address & string & GVL & Who should be cc'd on any tasks/forms that the server sends.
Example: form-cc-address = user2@host1.dom\\
\hline
full-bounce & boolean & GVL & Should bounces contain the full message or only the headers.
Example: full-bounce = false

Default value is false\\
\hline
mailserver & string & GV & The name of the outging SMTP server to use.
Example: mailserver = mail.host1.dom

Default value is localhost\\
\hline
max-rcpt-tries & integer & GV & How many times to attempt reading a RCPT TO: response.
Example: max-recpt-tries = 3

Default value is 5\\
\hline
send-as & string & GVL & Controls what the SMTP return path is set to.
Example: send-as = list2-bounce@test2.dom\\
\hline
sendmail-sleep & boolean & GV & Should we attempt to sleep a short time between message recipients.
Example: sendmail-sleep = on

Default value is false\\
\hline
sendmail-sleep-length & duration & GV & Override if you need the 'Sendmail Sleeper' option at a different
duration than the 3 second default. Example:  sendmail-sleep-length = 1 s

Default value is 3 s\\
\hline
smtp-blind-blast & boolean & GV & Should Ecartis perform all SMTP operations without regard for result
codes (e.g. trust that it was delivered)? NOT RECOMMENDED.
Example: smtp-blind-blast = yes

Default value is false\\
\hline
smtp-queue-chunk & integer & GVL & Maximum recipients per message submitted to the mail server. Larger lists will be split into chunks of this size.
Example: smtp-queue-chunk = 25\\
\hline
smtp-retry-forever & boolean & GV & Should Ecartis continue to wait for SMTP responses for an infinite amount of time, e.g. never give up? This can negatively impact delivery times. Example: smtp-retry-forever = yes

Default value is false\\
\hline
smtp-socket & integer & GV & Which socket should the SMTP server be contacted on.
Example: smtp-socket = 26

Default value is 25\\
\hline
\multicolumn{4}{l}{SUBSCRIBE OPTIONS}\\
\hline
Variable & Type & Valid & Description\\
\hline
an-sildmient-subscribe & boolean & GVL & If set true, when an admin subscribes a user to a list they
will receive no subscription notification AND no welcome message.
Example: admin-silent-subscribe = no

Default value is false\\
\hline
admin-subscribe-notice & choice & GVL & Notifications sent when an administrator (ADMIN) adds a subscriber.
The value of 'all' will send both the notification and welcome text (if configured) to the subscriber. The
'silent' option will send neither. Example:  admin-subscribe-notice = welcome

Default value is all\\
\hline
admin-unsubscribe-notice & choice & GVL & Notifications sent when an administrator (ADMIN) removes a subscriber.
The value of 'all' will send both the notification and goodbye text (if configured) to the subscriber. The
'silent' option will send neither. Example:  admin-unsubscribe-notice = goodbye

Default value is all\\
\hline
administrivia-include-requests & boolean & GVL & Should the mail which caused the (un)subscription action
be included in the message to the administrivia address. Example: administrivia-include-requests = on

Default value is false\\
\hline
goodbye-subject & string & GVL & Subject to be used in the mail sent if someone unsubscribes off a list.
Example: goodbye-subject = You have signed off '\textless\$list\textgreater'

Default value is You have signed off '\textless\$list\textgreater'\\
\hline
no-administrivia & boolean & GVL & Should the administrivia address be notified when a user subscribes or unsubscribes.
Example: no-administrivia = on

Default value is false\\
\hline
owner-fallback & boolean & GVL & Should the list-owner be used if administrivia-address is not defined when
notifying of (un)subscribes. Example: owner-fallback = true

Default value is false\\
\hline
prevent-second-message & boolean & GVL & If set 'true', then when someone tries to subscribe or unsubscribe
to an Ecartis list and no other commands are given, the 'Ecartis command results' message will not be sent.
Example: prevent-second-message = true

Default value is false\\
\hline
silent-resubscribe & boolean & GVL & Silence warnings when an already-subscribed user re-subscribes.
Example: silent-resubscribe = on

Default value is false\\
\hline
subscribe-confirm-file & string & GVL & File sent to someone subscribing to a list, if subscribe-mode = confirm.
Example: subscribe-confirm-file = text/subscribe-confirm.txt

Default value is text/subscribe-confirm.txt\\
\hline
subscribe-confirm-subject & string & GVL & Subject to be used in the mail sent to someone subscribing to a list, if subscribe-mode = confirm.
Example: subscribe-confirm-subject = Subscription confirmation for '\textless\$list\textgreater'

Default value is Subscription confirmation for '\textless\$list\textgreater'\\
\hline
subscribe-mode & choice & GVL & Subscription mode for the list.
Example: subscribe-mode = open

Default value is closed\\
\hline
subscription-acl & boolean & GVL & If 'true' and the file given in 'subscribe-acl-file' exists,
a subscription access list check will be performed when users attempt to subscribe to the list.
Example: subscription-acl = true

Default value is true\\
\hline
unsubscribe-confirm-file & string & GVL & File sent to someone unsubscribing from a list, if unsubscribe-mode = confirm.
Example: unsubscribe-confirm-file = text/unsubscribe-confirm.txt

Default value is text/unsubscribe-confirm.txt\\
\hline
unsubscribe-confirm-subject & string & GVL & Subject to be used in the mail sent to someone unsubscribing from a list,
if unsubscribe-mode = confirm. Example: unsubscribe-confirm-subject = Unsubscription confirmation for '\textless\$list\textgreater'

Default value is Unsubscription confirmation for '\textless\$list\textgreater'\\
\hline
unsubscribe-mode & choice & GVL & Unsubscription mode for the list.
Example: unsubscribe-mode = closed

Default value is open\\
\hline
welcome-subject & string & GVL & Subject to be used in the mail sent to new subscribers of a list.
Example: welcome-subject = Welcome to list '\textless\$list\textgreater'

Default value is Welcome to list '\textless\$list\textgreater'\\
\hline
\multicolumn{4}{l}{TEMPBAN}\\
\hline
Variable & Type & Valid & Description\\
\hline
tempban-default-duration & duration & GVL & If an administrator issues the tempban command without
a duration, this default will be used.
Example: tempban-default-duration = 7 d

Default value is 7 d\\
\hline
\multicolumn{4}{l}{TIMEOUTS}\\
\hline
Variable & Type & Valid & Description\\
\hline
adminreq-expiration-time & duration & GVL & How long until administrative request cookies expire.
Example: adminreq-expiration-time = 3 h\\
\hline
cookie-expiration-time & duration & GVL & How long until a generated cookie expires.
Example: cookie-expiration-time = 3 d 6 h

Default value is 1 d\\
\hline
filereq-expiration-time & duration & GVL & How long until config file request cookies expire.
Example: filereq-expiration-time = 3 h\\
\hline
modpost-expiration-time & duration & GVL & How long until a moderated post cookie expires.
Example: modpost-expiration-time = 2 h\\
\hline
reply-expires-time & duration & GV & How long until an automatic reply expires from the mailbox
Example: reply-expires-time = 3 h

Default value is 1 d\\
\hline
subscription-expiration-time & duration & GV & How long until subscription verification cookies expire.
Example: subscription-expiration-time = 5 d\\
\hline
unsubscription-expiration-time & duration & GV & How long until unsubscription verification cookies expire.
Example: unsubscription-expiration-time = 5 d \\
\hline
\multicolumn{4}{l}{TOLIST}\\
\hline
Variable & Type & Valid & Description\\
\hline
megalist & boolean & GVL & Should we process this list on-disk instead of in memory? This disables the
receipient list sorting and list-merging functionality of Ecartis, in order to prevent large memory
footprint operations. It is useful for lists where the receipient list is too large to effectively do
memory-based operations on. Example: megalist = true

Default value is false \\
\hline
no-loose-domain-match & boolean & GVL & Should the server treat users of a subdomain as users of
the domain for validation purposes. Example: no-loose-domain-match = on

Default value is false \\
\hline
per-user-modifications & boolean & GVL & Do we do per-user processing for list members.
Example: per-user-modifications = false

Default value is false \\
\hline
sort-tolist & boolean & GVL & Should the recipients be sorted by domain before sending. This
is memory expensive and is a bad idea if your SMTP server already does this sorting. If you SMTP
server doesn't do this, it can really improve outgoing mail performance.
Example: sort-tolist = off

Default value is true \\
\hline
tolist-send-pause & integer & GVL & How long (in milliseconds) do we sleep between SMTP chunks.
Example: tolist-send-pause = 30

Default value is 0 \\
\hline
\multicolumn{4}{l}{VACATION}\\
\hline
Variable & Type & Valid & Description\\
\hline
vacation-default-duration & duration & GVL & If a person sends the vacation command without a
duration, how long they will be set vacation. Example: vacation-default-duration

Default value is 14 d \\
\hline
\end{longtable}
\end{document}

