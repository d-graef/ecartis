\documentclass{book}
\usepackage{ecartis}
\usepackage{hthtml}
\usepackage{verbatim}

\begin{document}

\frontmatter

\title{Ecartis \\ Modular Mailing List Manager \\ http://www.ecartis.org/}
\author{Copyright \copyright\ 1998--2002 Rachel Blackman, JT Traub and contributors.}
\maketitle

\chapter{Version History}
\label{version}

\begin{description}
	\item[2002-04-29] Rewrite from Listar docs to Ecartis docs; added
                      paragraph to Introduction about name change from Listar
                      to Ecartis
\end{description}

\chapter{Notes About This Document}

The Ecartis manual is most definitely a work in progress.  As is common with
many software projects, development of the software has far exceeded
development of the documentation to explain it; this is a shortcoming we are
attempting to address.  

Discussion of this documentation should be directed to the mailing list 
\htmailto{ecartis-doc@ecartis.org}.  Subscription information for the list is
available at \hturl{http://www.ecartis.org}.  Anyone who has submissions they
would like added to the documentation, or has suggestions for rewording,
changes, etc. to the existing documentation should direct their comments to
this list.

For purposes of portability, this documentation is currently maintained in
\LaTeX, a very simple, yet powerful text markup language that provides us the
ability to easily generate versions of this documentation formatted as
PostScript, PDF, HTML, or raw text.  Those wishing to make direct changes to
the documentation source should first familiarize themselves with \LaTeXe,
and should keep the ideal of portability in mind when making decisions
regarding format.



\tableofcontents



\mainmatter


\chapter{Introduction}
\label{int}
\section{What is a Mailing List Manager?}
\label{int:whatis}

A mailing list manager is a piece of computer software which accepts a piece
of e-mail from a single source and distributes it to a number of recipients.
Possible uses for such a piece of software include a monthly newsletter for
customers of a business, a way to distribute information to users of a
particular software package --- for example, notification of a security fix
--- or a way for people who share a common interest to communicate with each
other.
   
How people choose to implement mailing lists can vary widely.  The simplest
method involves setting up a single address on an e-mail domain you control,
and forwarding it to a number of people.  This has a number of disadvantages,
however; users have no way to add themselves to the list, or remove themselves
from it, there are no ways to restrict who can send mail to a distribution
list, and other such headaches.
   
Most people, therefore, choose to use a piece of software to manage such lists
for them; hence the term mailing list manager, or MLM for short.  Many people
also refer to such packages as `list-servers,' as they are server software for
managing lists.  There are an ever-increasing number of MLMs available out
there, but almost all share certain common traits; the ability for users to
subscribe or unsubscribe themselves from a distribution list, the ability for
an administrator to manually remove a user, the ability to restrict posting to
a small number of individuals, and so on.  In addition, some MLMs support many
additional advanced features, such as the ability to filter out unsolicited
commercial e-mail (UCE, also known as `spam').
   
\section{A Brief History of MLMs}
\label{int:history}

Back in the mid-1980's, the system of interconnected computers we know as the
Internet was not yet around.  While in the United States, there was some
interconnection between colleges and the government's ARPAnet, the only way
any other machines --- such as those at different universities ---
communicated was over a system called BITNET.  BITNET machines
let messages for each other pile up, and then would call each other over the
phone lines and send the messages.

BITNET had a central control post, a Network Information Center (or NIC)
called `BITNIC.'  BITNIC kept a number of distribution lists for BITNET users.
However, the BITNIC's lists were set up in the primitive way mentioned in
Section~\ref{int:whatis}; a single address with no way for users to add
themselves or remove themselves.  If you wished to be on a mailing list, you
had to contact the BITNIC staff and have them add you by hand.

Unfortunately, as BITNET grew larger, managing the lists by hand was no longer
feasible.  Additionally, since all mail for BITNIC was affected by the traffic
of the lists --- which by now were quite large --- even private BITNET e-mail
was affected.  It may be hard for users of today's Internet to imagine, but
try to picture things becoming so slow that when someone sent you an e-mail,
it took over a week to arrive in your mailbox.  Clearly, something needed to
be done.
   
Since the source of the problem was the traffic on BITNIC's mailing lists, a
computer science student named Eric Thomas decided to write a piece of
software to replace the manually-managed mailing lists.  It also used a number
of alternate paths to send e-mail to the list recipients, to keep it from
clogging BITNET's mail pathways, but what was more important to the future of
MLMs was the fact that this software allowed users to add and remove
themselves from the lists, instead of relying on a BITNIC system administrator
to do it for them.  When it went online in July 1986, a piece of software was
managing a mailing list for the first time --- Eric Thomas had created the first
MLM.  Soon thereafter, others created similar packages for other systems, such
as the LISTSERV imitation for UNIX, Listproc.
   
As time went on, Eric Thomas' LISTSERV developed into a commercial product
beyond BITNET and was ported to other systems, and is still widely used on the
Internet today.  However, as the days of BITNET faded into the past and the
Internet became a reality, students who wished to run their own lists and did
not have access to the funds necessary to purchase a license for LISTSERV
began to look at developing their own MLMs to meet their particular needs.
   
Perhaps one of the most popular is Majordomo, which has been
worked on by a variety of people over the years.  Majordomo is written in the
Perl scripting language, which is perhaps its greatest failing as it makes
Majordomo rather inefficient.  However, Perl is very powerful for text
processing, and thus Majordomo is readily extendible by those who know Perl
and are willing to learn Majordomo's source code.
   
A good --- if somewhat biased --- summary of the history of MLMs is available
online from Lyris Technologies (who are themselves the authors of a commercial
MLM called Lyris, which is targeted specifically at business users) at
\hturl{http://www.lyristechnologies.com/historyls.html}
   
\section{History of Ecartis}
\label{int:history:Ecartis}

Ecartis was born as a simple little project called `uList' (microList) in
October 1997.  The original author, Rachel Blackman, had been using the
Majordomo mailing list package but wished for one that was more efficient and
did not require any special system permissions to run.  Additionally, Rachel
wanted a server that would allow the individual subscribers to change their
subscription options --- one of the more desirable features of LISTSERV.
   
The original design for uList was simple enough; users needed to be able to
set a few simple options on themselves as well as perform all the standard
operations that most MLMs provide.  However, Rachel got tired of having to
constantly rewrite the core processing code as new functionality was added,
and changed uList over to a modular system, where almost all of the system was
added to a changeable table of information.  Suddenly, the system could be
easily changed; a new command could be added with only a few lines of code, or
a new step in processing a message to be sent.  The small uList program
suddenly had more potential.  And to go with the new design, the software got
a new name: Listar.
   
Listar developed into a more stable piece of software, and a mailing list was
set up using it called listar-dev, for those who were interested in the
ongoing development of the project.  On January 12, 1998, Joseph (JT) Traub
began to work on the project as well.  JT's first contribution to the project
was the development of a `dynamic module' system.  Since Listar was already
based entirely around a dynamic model, this allowed new Listar plugin modules
(or LPMs) to be installed and immediately provide new commands, subscription
flags, or functionality.
   
The first public release of Listar came in February of 1998, and it was used
only by a few curious parties.  However, many provided good feedback on
features and functionality they wished to see in such a project, and Listar
grew rapidly into a more mature program.
   
Then, tragedy struck.  In October of 1998, the machine that Listar's main
development resources were housed on was cracked into by a malicious
individual, and the mailing lists were destroyed.  The Listar source code
remained safe in backup copies, but the lists themselves were no longer
available.  The recovery from this event took a while, and development on
Listar was slow again afterwards at first, until users discovered the site was
back and resubscribed to the lists.

Once past the recovery, however, 1999 proved to be a year full of rapid
development for Listar.  It became even further fleshed out, and began to be
used by some large organizations, such as the Internet Software Consortium.
The developers eagerly accepted suggestions and created new LPMs to add custom
functionality, while folding additional functionality into the core module.

Listar encountered one other setback late in 2000 (again in October) when
Rachel received a Cease \& Desist letter from the company holding a trademark
for ListSTAR, a defunct MLM for Apple computers.  Despite the obvious
differences in the names (Listar being ``List'' in Spanish; ListSTAR being
List + Star) no agreement could be reached, and in mid-2001 a name change for
the project was announced: Ecartis.  

As of this writing, Ecartis approaches the fifth anniversary of its birth as
uList, and has a growing user base as well as a budding external developer
community.  Some of the features that appeared first in Ecartis are now being
borrowed for other MLMs.  Ecartis is an Open Source project, so users are
encouraged to join in creating code for it.
   
\chapter{Installation}
\label{install}

\section{Introduction}
\label{install:intro}

There are several ways that Ecartis can be obtained, and thus several methods
of installation.  Perhaps the most common is as a collection of source code;
which will compile on any number of UNIX-type platforms, as well as Microsoft
Windows 95/98 or NT.
   
The package can also be obtained as an RPM (Redhat Package Manager)
installation, either from the Ecartis FTP site, or from RedHat's PowerTools
collection or the RedHat Contrib|Net.  It can also be obtained as a .deb file
for Debian GNU/Linux from the Ecartis FTP site or a local Debian mirror.  Or
lastly, for those more inclined towards living on the edge of development, the
source repository for Ecartis is available for read via CVS.
   
This document makes the assumption that if you choose to download either the
.rpm or .deb versions of Ecartis to install, you already know how to use that
package manager to install it, and will thus only cover compiling from source
tarball, and compiling under Windows.

\section{Source Tarball Installation}
\label{install:tarball}

The Ecartis source tarball will unpack into a directory called ecartis-version,
where version is the version of the tarball you are unpacking.  Fairly
straightforward.  Feel free to explore the directory tree before moving along
to the installation instructions for Unix (Section~\ref{install:unix}) or
Windows (Section~\ref{install:windows}).

\section{CVS Installation}
\label{install:CVS}

CVS stands for `Concurrent Versions System,' and is a method of allowing
multiple programmers to work on source code simultaneously, from a single
repository of source code. The Ecartis installation is designed to be able to
work from a CVS installation, to make it easy to keep an installation
up-to-date.  To access the Ecartis CVS tree, you need a CVS client capable of
CVS-pserver access.
   
Find the location that you wish to have your `ecartis' tree, and enter the
following command:

\begin{quote}
\footnotesize
\begin{verbatim}
cvs -d :pserver:guest@cvs.ecartis.org:/usr/cvsroot login
\end{verbatim}
\end{quote}

When it prompts you for a password, enter guest.  Then type:
   
\begin{quote}
\footnotesize
\begin{verbatim}
cvs -d :pserver:guest@cvs.ecartis.org:/usr/cvsroot checkout ecartis
\end{verbatim}
\end{quote}
   
You will see a list of filenames scroll by as CVS retrieves the current Ecartis
source tree, and then you will have a source tree as if you'd unpacked it from
the source tarball, except that it has a `CVS' subdirectory in each directory;
this allows your installation to remain synched with CVS.  Now, any time you
wish to get the latest version, go into the Ecartis directory and do:
   
\begin{quote}
\footnotesize
\begin{verbatim}
cvs update -P -d
\end{verbatim}
\end{quote}

For information on CVS and CVS servers, visit Cyclic Software, the authors of
CVS, at \hturl{http://www.cyclic.com/} --- there are graphical CVS clients
available for several operating systems as well as binaries for various
systems.
   
\section{Compiling Under UNIX}
\label{install:unix}

To compile under a UNIX-style system (such as Linux, FreeBSD, SunOS, etc.),
Ecartis needs GCC or EGCS.  While it is technically possible that Ecartis would
compile under the GCC for Windows that Cygnus provides, it is unlikely.
Ecartis has a specific Windows port; see Section~\ref{install:windows} for more
information.

The first step under UNIX is to go into the `src' directory under where you
unpacked the tarball (or did a CVS checkout to).  Copy Makefile.dist to
Makefile.  Now you'll want to pull Makefile up in your favorite text editor.
There are a variety of comments about what you'll need to tune for specific
operating systems; tweak the file as appropriate for your operating system and
then save it.  Now type `make'.
   
Ecartis uses the GNU Make program to handle the build process.  Some operating
systems, like Linux, provide this as the default `make' program, while others,
like FreeBSD, have it available as `gmake'.  If you have BSD Make, when you
run `make' in the Ecartis src directory, you will get a string of errors that
look like:

\begin{quote}
\footnotesize
\begin{verbatim}
"Makefile", line 114: Need an operator
"Makefile", line 116: Need an operator
"Makefile", line 118: Need an operator
"Makefile", line 129: Need an operator
\end{verbatim}
\end{quote}

If this is the case, you will need to run `gmake' instead of `make'.  If the
build process is going correctly, you will see output like this:
   
\begin{quote}
\footnotesize
\begin{verbatim}
gmake[1]: Entering directory `/home/loki/src/ecartis/src/modules/bounc
er'
gcc -fPIC -DDYNMOD -Wall -Werror -I../../inc -I. -DGNU_STRFTIME   -c 
bouncer.c
gcc -shared   -o bouncer.lpm bouncer.o
cp bouncer.lpm ../../build
gmake[1]: Leaving directory `/home/loki/src/ecartis/src/modules/bounce
r'
[Build: bouncer] Built module successfully.
gmake[1]: Entering directory `/home/loki/src/ecartis/src/modules/lista
rchive'
gcc -fPIC -DDYNMOD -Wall -Werror -I../../inc -I. -DGNU_STRFTIME   -c 
archive.c
gcc -shared   -o ecartischive.lpm archive.o
cp ecartischive.lpm ../../build
gmake[1]: Leaving directory `/home/loki/src/ecartis/src/modules/ecartis
chive'
[Build: ecartis] Built module successfully.
\end{verbatim}
\end{quote}

\ldots and so on.  The actual text of your output may vary slightly depending
on the system.  Assuming no build errors occur, you should end up with a final
line that looks something like this:
   
\begin{quote} 
\footnotesize
\begin{verbatim}
gcc   -o ecartis alias.o command.o user.o parse.o list.o core.o forms.
o smtp.o io.o regerror.o regsub.o regexp.o flag.o cookie.o file.o mod
ule.o fileapi.o variables.o internal.o cmdarg.o modes.o dynmod.o unmi
me.o codes.o hooks.o tolist.o mystring.o lma.o userstat.o snprintf.o 
moderate.o mysignal.o unhtml.o liscript.o submodes.o lcgi.o upgrade.o
\end{verbatim}
\end{quote}

   
If there are no linker errors, you now have all the binaries you need.  Type
`make install' (or `gmake install') and it will place all the Ecartis binaries
into the appropriate places.  If you wish to run Ecartis directly from this
location, it is now ready to go.  Otherwise, you need to copy things
elsewhere.  If you are using the dynamic module mode (Ecartis's recommended
configuration), there is a directory under where you unpacked Ecartis which is
called `modules', and contains a number of files with an .LPM extension.  And
in the directory where you unpacked Ecartis is a binary executable called
ecartis.  The ecartis binary should be placed wherever you intend to have your
Ecartis installation, along with the ecartis.cfg file.  The modules should be
placed into a modules directory, which by default is a subdirectory called
modules under wherever the ecartis binary is.  It is possible to split up
Ecartis into a number of separate directories, however, as the Debian
installation does.  At this point, you should have enough information to move
along to Section~\ref{starting}, Getting Started.
   
\section{Compiling Under Windows}
\label{install:windows}

Ecartis can be compiled under Windows, but there are a few caveats.  First, it
still functions as a mail filter, so you need a way to feed the mail to it;
many Windows mail servers do not function in this way.  Secondly, it requires
Microsoft Visual C++ 5.0 or higher to compile.

If these still meet your criteria, Ecartis is fairly easily compiled.  When you
unpack the source tree, or do a CVS checkout, you will want to go to the
top-level directory, and load the ecartis.dsw file.  Then you can simply go and
do a batch build of all the targets.  In the end, you will have a debug and a
release ecartis.exe, in \verb+src\debug\Ecartis.exe+ and
\verb+src\release\Ecartis.exe+.  You will also have \verb+src\modules\debug+
and \verb+src\modules\release+, which will contain debug and release builds of
all the LPM modules.  From here on out, Ecartis is configured like any of the
UNIX installations.
   
There is also a commercially supported mailing list package for Windows which
is based on Ecartis.  It is called SLList and is sold and supported by Seattle
Lab, at \hturl{http://www.seattlelab.com/}.

\chapter{Getting Started}
\label{starting}
   
\section{Introduction}
\label{starting:intro}

Most MLMs function as a mail filter --- in other words, a program that is
invoked when a piece of mail arrives for a specific address, and which is fed
the mail that arrived.  How individual mail servers function with this varies;
the widely-used package sendmail uses a file called aliases to determine
things like this, while other packages such as qmail handle it differently.
We'll cover a few generic pieces, and then a few specific mail servers.
   
\section{Ecartis and Mail Filters}
\label{starting:filters}

A list running on Ecartis has several addresses associated with it.  The list
address itself (for example, foolist@domain.com), the list administrators (for
example, foolist-admins@domain.com), and so on.  In addition, Ecartis has an
address for itself (ecartis@mydomain.com).  Each of these addresses is
represented by an alias.  The ecartis@mydomain.com address calls the Ecartis
program without any parameters, while the individual list ones have various
parameters.
   
Hence, when you set up the Ecartis aliases for your system, you'll probably
have a number of things looking like this:
   
\begin{quote}
\footnotesize
\begin{verbatim}
# Ecartis address
ecartis:           |/home/ecartis/ecartis
# Testlist aliases
testlist:         |"/home/ecartis/ecartis -s testlist"
testlist-request: |"/home/ecartis/ecartis -r testlist"
testlist-admins:  |"/home/ecartis/ecartis -admins testlist"
testlist-repost:  |"/home/ecartis/ecartis -a testlist"
testlist-bounce:  |"/home/ecartis/ecartis -bounce testlist"
\end{verbatim}
\end{quote}

This means that when some e-mail comes in for the `ecartis' address, the
mailserver would run the program /home/ecartis/ecartis and pass the contents of
that e-mail to the program.  The `testlist' address, for example, will also
run the program, but it will have some additional parameters.  If the aliases
look a bit confusing, don't worry; Ecartis will generate them for you when you
make a new list, but if your mailserver isn't sendmail, you may have to edit
the way the aliases look slightly.  Most mailservers use an alias format
similar --- or identical --- to the sendmail format, so for most servers you
will be able to use the aliases that Ecartis generates directly.
   
\subsection{Sendmail}
\label{starting:filters:sendmail}
   
Sendmail (\hturl{http://www.sendmail.org}) is perhaps the widest used mail
server on the Internet, and comes preinstalled on most UNIX-type systems.
Sendmail is extremely configurable --- almost too configurable, as it is
possible to get lost in the configuration options --- and is more than capable
of using Ecartis as a mail filter.
   
There are several ways to handle setting up Ecartis under Sendmail.  Perhaps
the easiest is simply to copy and paste the aliases that you get from a Ecartis
newlist.pl script into the Sendmail /etc/aliases file, and run the newaliases
program.  Others may prefer to create a separate aliases file for Ecartis; the
method to do this depends on how you're setting up your sendmail config file.
The direct method is to go into the /etc/sendmail.cf file, find the line
referring to /etc/aliases, and add AliasFile line below it for the Ecartis
aliases file.

Perhaps the most common pitfall of people setting up new Ecartis installations
--- or indeed, any new mail filter under Sendmail --- is the fact that many
Sendmail installations come with a program called SMRSH already enabled.
SMRSH, or the SendMail Restricted SHell, is a program that increases system
security by making sure that Sendmail only allows programs in a certain
directory to be run as mail filters.  Where this directory is varies depending
on your Sendmail installation, but a common one is /etc/smrsh.
   
If any attempt to mail Ecartis meets with a bounce message like the following,
you are using SMRSH.
   
\begin{quote}   
\footnotesize
\begin{verbatim}
----- The following addresses had permanent fatal errors -----

"|/home/ecartis/ecartis -s mylist"

     (expanded from: <mylist@mydomain.com>)


    ----- Transcript of session follows -----

sh: ecartis not available for sendmail programs

554 "|/home/ecartis/ecartis -s mylist"... Service unavailable
\end{verbatim}
\end{quote}
   
The solution is to locate the directory that SMRSH is using for its programs,
and write a shell script like this, placing it there.
   
\begin{quote}
\footnotesize
\begin{verbatim}
#!/bin/sh
/home/ecartis/ecartis $@
\end{verbatim}
\end{quote}

Of course, /home/ecartis should be the path where you installed the Ecartis
binary.  It is also vitally important to make sure that this wrapper script is
set executable so that smrsh can run it.  Also make sure that the directory
leading up to the Ecartis binary has execute permissions for whatever user or
group your mail server runs as.
   
Then you'd want to go and change the aliases so that they had the path to the
shell script wrapper.  For example, changing \ldots
   
\begin{quote}   
\footnotesize
\begin{verbatim}
mylist:           |"/home/ecartis/ecartis -s mylist"
\end{verbatim}
\end{quote} 
   
\ldots to \ldots
   
\begin{quote}
\footnotesize
\begin{verbatim}
mylist:           |"/etc/smrsh/ecartis -s mylist"
\end{verbatim}
\end{quote}
   
\ldots then Sendmail should allow Ecartis to be used as a mail filter.
   
\subsection{Exim}
\label{starting:filters:exim}

Exim is another of the mail servers out there, though it doesn't come
preinstalled on many --- if any --- systems.  Information on it can be found
at \hturl{http://www.exim.org/}.  Exim was developed at the University of
Cambridge over in England.  To make Ecartis work with Exim, you could take the
simple approach like with Sendmail and paste the Ecartis aliases into the
existing Exim aliases file, or you can go into exim.conf and add the following
lines:
   
\begin{quote}
\footnotesize
\begin{verbatim}
ecartis_aliases:
  driver = aliasfile
  file_transport = address_file
  pipe_transport = address_pipe
  file = /usr/lib/ecartis/aliases
  search_type = lsearch
  user = ecartis
  group = ecartis
\end{verbatim}
\end{quote}
   
Other than that, Ecartis should function correctly with Exim out-of-the-box.
   
\subsection{Postfix}
\label{starting:filters:postfix}
   
Postfix (\hturl{http://www.postfix.org}) is another mail server, designed with
the goal of creating a package as secure and fast as sendmail, while still
providing as much backwards compatibility as possible.  For Postfix
installations, you can once again take the simple route and paste the Ecartis
aliases into the default aliases file and then run the postaliases program any
time you change it.
   
If you wish to have a separate Ecartis aliases file, you will need to pull
/etc/postfix/main.cf up in your favorite text editor.  Find alias\_maps line,
and add the ecartis aliases file to it.  The resulting line will look
something like:
   
\begin{quote}   
\footnotesize
\begin{verbatim}
alias_maps = hash:/etc/aliases, hash:/etc/mail/ecartis.aliases
\end{verbatim}
\end{quote}
   
Then run postalias /etc/mail/ecartis.aliases every time you add aliases to the
Ecartis file.  Of course, /etc/mail/ecartis.aliases should be replaced by the
path to where you keep your Ecartis aliases file.
   
\subsection{qmail}
\label{starting:filters:qmail}

The qmail (\hturl{http://www.qmail.org}) program is perhaps the second most
widely used UNIX mail server on the Internet, being considered to be small,
fast and secure.  However, the method of setting up qmail aliases is far
different from setting up aliases under any other mail package.  Instead of a
single file that contains multiple aliases mapping through to Ecartis, you
need to create what are called dot-qmail files in the home directory of the
qmail system user (often `alias').
   
If the line in a normal aliases file would be \ldots

\begin{quote}
\footnotesize
\begin{verbatim}
mylist:           |"/home/ecartis/ecartis -s mylist"
\end{verbatim}
\end{quote}
   
\ldots then you would need to create a file in the home directory of the qmail
system user.  This file would be named .qmail-mylist and which will contain
the text \verb+|"/home/ecartis/ecartis -s mylist"+
   
You could create such a file by going to the appropriate directory and typing:
   
\begin{quote}
\footnotesize
\begin{verbatim} 
echo "|/home/ecartis/ecartis -s mylist" > .qmail-mylist
\end{verbatim}
\end{quote}

In other words, for a line like \ldots

\begin{quote}   
\footnotesize
\begin{verbatim}
address:          command
\end{verbatim}
\end{quote}   
   
\ldots you want \ldots
   
\begin{quote}
\footnotesize
\begin{verbatim}
echo "command" > .qmail-address
\end{verbatim}
\end{quote}   
   
Unlike most mail servers, you do not need to run a program to rebuild aliases
under qmail.
   
Ecartis's internal newlist command can create dot-qmail files for a given
list; to set up your Ecartis installation to do so by default, pull
ecartis.cfg up in your favorite text editor, and add the following line:

\begin{quote}
\footnotesize
\begin{verbatim}
newlist-qmail = yes
\end{verbatim}
\end{quote}
   
Then, when you create a list, it will have a subdirectory called
qmail-aliases, and in that directory will be all the dot-qmail files you need
to copy over to your qmail global aliases directory.
   
You can force this feature on by adding the command-line argument -qmail when
creating a new list.
   
It is also worth noting that there is an optional add-on package for qmail by
Dan Bernstein called FastForward
(\hturl{http://www.pobox.com/~djb/fastfoward.html}) which allows qmail to use
/etc/aliases.
   
\chapter{Configuring the Server}
\label{configuring}

\section{Primary (Global) Config File}
\label{configuring:global}

   
\chapter{Setting Up a List}
\label{newlist}

\section{Creating the List}
\label{newlist:create}

Ecartis contains a set of commands internally to make it easier to create a
list. This entire process can be invoked from the command line and easily
controlled, and the various installed modules are responsible for generating
the configuration file and documenting it, as well as any other custom
handling for new list creation.
   
To begin this process, you will want to run the Ecartis binary with the
command-line argument -newlist; this takes one parameter, the name of the new
list. In addition, if you're creating a list for a virtual host, simply put
the virtual host config file at the beginning of the line. Ecartis determines
what path to put in the aliases by what path you use when invoking it, so in
this case be sure to run it with a full path name. Here are two examples:

\begin{quote} 
\footnotesize
\begin{verbatim}
/home/ecartis/ecartis -newlist testlist
\end{verbatim}
\end{quote}

\begin{quote}
\footnotesize
\begin{verbatim}
/etc/smrsh/ecartis -c virthost1.cfg -newlist testlist2
\end{verbatim}
\end{quote}
   
Ecartis will prompt you for the e-mail address of the list administrator; this
should be an address from which mail can be sent as well as received. Do not
use a forwarding address for the primary administrator of the list. Once
you've entered this information, the script will create the basic list
configuration and basic list directory. Then it will give you a block of
information that looks something like this:
   
\begin{quote}   
\footnotesize
\begin{verbatim}
# Aliases for list 'testlist'
testlist: "|/home/ecartis/ecartis -s testlist"
testlist-request: "|/home/ecartis/ecartis -r testlist"
testlist-repost: "|/home/ecartis/ecartis -a testlist"
testlist-admins: "|/home/ecartis/ecartis -admins testlist"
testlist-moderators: "|/home/ecartis/ecartis -moderators testlist"
testlist-bounce: "|/home/ecartis/ecartis -bounce testlist"
testlist-owner: joe@mydomain.com
\end{verbatim}
\end{quote}
   
You will want to put this information into your mail server's alias file, and
you may need to run a program to rebuild the aliases database (for example,
Sendmail users will need to run the newaliases program).
   
As a side note for advanced users, all the output Ecartis gives in this mode
OTHER than the aliases is on stderr, while the aliases are printed to stdout,
thus making it easy to simply redirect output and append it to your aliases
file.
   
If you have Ecartis set up to use qmail instead (see
Section~\ref{starting:filters:qmail}), you will not get the block of aliases,
but will instead find that in the directory of your new mailing list is a
subdirectory called qmail-aliases; in this directory are the dot-qmail files
you will want to copy over to your qmail global aliases directory.
   
\section{Editing the List Configuration}
\label{newlist:configure}
   
Now, you doubtless want to go tune a few of the default settings in the list
configuration file.  Go to the directory where you have your Ecartis lists
stored, and go into the directory with the name of the list you just made.
Bring the config file up in your favorite editor, and edit to taste.  There is
an appendix of configuration variables at the end of this document.
   
   
This section needs more detail.
   
Once you have things configured to your taste, you're ready to move on to your
first interaction with Ecartis!
   
\chapter{Interacting with Ecartis}
\label{interface}
   
\section{Concepts}
\label{interface:concepts}
   
Unlike many MLMs, Ecartis has something called a `list context.'  What this
means is that, much like someone in a conversation, Ecartis remembers what the
previous command in a session referred to.  If you tell it to do a command on
list1 and then don't give it a list name for the next command, it will assume
you are still talking about list1.  If, however, you tell it to refer to list2
in the second command, it will do so.  How this works will become clearer as
we move along.
   
Ecartis also has what is called a `secure' mode.  Certain commands --- such as
administrative functions --- must be performed through a method that prevents
people from pretending to be you.  Some MLMs only check the address a message
comes from before allowing administrative commands, leaving them open to easy
hacking by a mail-spoofer.  Other MLMs use passwords, which could be
discovered.  Ecartis uses a unique method which secures it in any situation
but that where either the machine that Ecartis runs on has been compromised,
or the e-mail account of an administrator has (something out of the realm of
control of an MLM).  This method is referred to as using `cookies.'
   
Ecartis's cookies are used for more than just administrative modes, as well.
When a cookie is ``baked,'' (created by Ecartis) it has a specific ``flavor.''
A cookie of one flavor cannot be used for a task requiring a different flavor
--- for example, a cookie for a user to confirm their subscription could not
be used to authorize administrative commands.  If a cookie is used, it is
``eaten'' and cannot be used again.  If a cookie remains unused, eventually it
goes stale and is removed from the cookie jar.  And lastly, a cookie can be
baked for a specific person, and then only that person can use it.
   
It is the cookie system that allows Ecartis to have much of the security and
power it does.
   
\section{Normal Mode}
\label{interface:normal}

Normal mode is how the majority of users interact with Ecartis.  This is where
you send e-mail to the Ecartis server itself with a set of commands.  For
example, a user might send a message like this \ldots
   
\begin{quote}   
\footnotesize
\begin{verbatim}
Date: Tue, 28 Sep 1999 23:26:32 -0700 (PDT)
From: Joe Q. User <joe@someisp.com>
To: ecartis@mydomain.com
Subject:

lists
which
end
\end{verbatim}
\end{quote}
   
And Ecartis would send back a message like \ldots
   
   
\begin{quote}
\footnotesize
\begin{verbatim}
Date: Tue, 28 Sep 1999 23:27:27 -0700 (PDT)
From: Ecartis <ecartis@mydomain.com>
To: joe@someisp.com
Subject: Ecartis command results: lists

>> lists
Ecartis lists available on this machine:

testlist
          A test list.

>> which
Retrieving list subscriptions.

joe@someisp.com is subscribed to the following lists:
      testlist

>> end
Command set concluded.  No further commands will be processed.

---
Ecartis v0.126a - job execution complete.
\end{verbatim}
\end{quote}   
   
In other words, each line of the message is parsed by Ecartis, which tries to
find a valid command in it.  If there is a valid command in the subject line,
it will use that instead of parsing the body (unless configured to ignore the
subject line).  What commands are available vary depending on the Ecartis
installation, since an LPM can add new commands.
   
Additionally, if you enclose what is called a ``job/eoj wrapper,'' Ecartis
will ignore everything except what is between the beginning and ending of the
wrapper.  (Multiple job/eoj wrappers can be used per message.)  This is useful
for administrative functions, as you will see later.
   
\begin{quote}
\footnotesize
\begin{verbatim}
// job
commands
// eoj
\end{verbatim}
\end{quote}
   
   
When a command takes a list as one of the parameters, it turns that into the
list `context,' and if you omit a list from the next command, it will assume
you mean the same list as before.  When a command finally has another list
given, it will change the context for following commands.  Whenever a command
changes the list context, you will be told in the results list:
   
\begin{quote}
\footnotesize
\begin{verbatim}
List context changed to 'testlist' by following command.
>> who testlist
Membership of list 'testlist':
        joe@mydomain.com (ADMIN)
        jane@mydomain.com (ADMIN)
        joe@someisp.com

>> stats
Current account flags for 'joe@mydomain.com' on 'testlist':

        REPORTS
        CCERRORS
        ECHOPOST
        MODERATOR

List context changed to 'testlist2' by following command.
>> stats testlist2
Current account flags for 'joe@mydomain.com' on 'testlist2':

        ADMIN
        SUPERADMIN
        ECHOPOST
        REPORTS
        CCERRORS

>> who
Membership of list 'testlist2:
        joe@mydomain.com (ADMIN)
        joe@someisp.com
\end{verbatim}
\end{quote}
   
In the above example, the user (joe@mydomain.com) sent the commands:
   
   
\begin{quote}
\footnotesize
\begin{verbatim}
who testlist
stats
stats testlist2
who
\end{verbatim}
\end{quote}
   
Now you know how to issue commands to Ecartis, but what commands can you
issue?  Well, there are a great many of them, and they vary from installation
to installation because of customization.  Luckily, there is a way to query
Ecartis for what commands it supports.  If you send Ecartis the command
commands, it will send back a list of what commands it supports on a given
installation.  Some commands are only ever useful in messages Ecartis
generates, such as for secured-mode messages.
   

\section{Secure (Administrator) Mode}
\label{interface:admin}

Obviously, there needs to be a way to issue administrative commands to
Ecartis; it isn't feasible to expect every user to have access to the files on
disk that control a list they might potentially be an administrator on.  But
it is equally obvious that it's undesirable to have random unauthorized users
able to issue administrative commands.  Clearly, some sort of authentication
is needed.  Ecartis achieves this authentication with cookies, as mentioned
before.
   
There are two ways to handle administrative commands, but they have the same
end result.  The first method is to send Ecartis the command admin list.  If
you are a valid administrator for list, Ecartis will send you what is called
an admin wrapper.  This is something that looks like:
   
\begin{quote}
\footnotesize
\begin{verbatim}
// job
adminvfy testlist 37F1C6FC:58BB.1:ybxvznvfbabgnxharg
adminend
// eoj
\end{verbatim}
\end{quote}
   
You want to fill out any administrative commands (things like getconf, or
setfor, which require administrative permissions) between the adminvfy and
adminend lines, and send the wrapper back to Ecartis.
   
The second method was added in a later version of Ecartis, and is more
convenient in most cases.  Instead of sending the command admin list, send the
command admin2 list, and then give it the commands you want in the wrapper
followed by adminend2.  That will send back the wrapper already filled out,
and you can simply hit `reply' to approve it.  This is also where job/eoj
blocks come in handy!  Picture receiving a note from a user (say,
joe@someisp.com) which says that they need to be unsubscribed and can't
remember how.  You could simply reply and carbon-copy Ecartis, saying
something like:
   
\begin{quote}   
\footnotesize
\begin{verbatim}
> Hey, sorry... I know I should know how to unsubscribe myself
> But...
> Could you unsubscribe me for me?

Sure!

// job
admin2 testlist
unsubscribe joe@someisp.com
adminend2
// eoj
\end{verbatim} 
\end{quote}
   
Then you would receive back an already filled-out administrative wrapper such
as:
   
   
\begin{quote}
\footnotesize
\begin{verbatim}
// job
adminvfy testlist 37F1C6FC:58BB.1:ybxvznvfbabgnxharg
unsubscribe joe@someisp.com
adminend
// eoj
\end{verbatim}
\end{quote}
   
   
Ecartis can parse reply-formatted messages in many cases (the exceptions being
the putconf command and moderated messages), so when you use the admin2
command, you can simply hit reply to the wrapper it generates.
   
Any commands that are listed with an (ADMIN) after them in the commands list
that Ecartis will generate will only work between an adminvfy line and an
adminend line.  Normal commands will also work between those lines, but some
have different uses; for example, the stats command normally functions as
stats list, to allow a user to view statistics on themselves for a list.  But
in admin mode, you are locked to a specific list - the list you validated
administrator permissions for - so the command you would enter instead is
stats user, to allow an administrator to view stats for a specific user on
that list.
   
\section{Web Interface}
\label{interface:lsg2}

Information on LSG/2 needs to go here (after LSG/2 is finished).
   
\chapter{List Moderation}
\label{moderation}

Information on moderating mailing lists.
   
\chapter{Modules}

   
Will give a listing of the default Ecartis modules (administrivia, etc.) and
what they provide.
   
   
\chapter{Example List Configurations}

Some useful list configurations, like an announcement list or a connected set
of lists like the nwcpp-discuss and nwcpp-announce lists, or the
ecartis-support/ecartis-dev/ecartis-announce setup.
   
   
\chapter{Writing Modules}

Instructions on how to write a Ecartis module.


% \backmatter


\appendix
\chapter{Config Variables}
\label{config}

\end{document}
